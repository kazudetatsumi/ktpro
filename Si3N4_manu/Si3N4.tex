%#BIBTEX bibtex Si$_3$N$_4$
\documentclass[twocolumn,amsmath,amssymb,a4paper,prb,superscriptaddress,floatfix]{revtex4-1}
\usepackage[dvipdfmx]{graphicx}
\usepackage{natbib}
\usepackage{multirow}
\usepackage{amsmath}
\usepackage{bm}
\usepackage{mathrsfs}
\usepackage{url}
\usepackage{color}
\usepackage{ulem}
\begin{document}

\title{First-principles calculation of the lattice thermal
conductivities of $\alpha$-, $\beta$-, and $\gamma$-Si$_3$N$_4$}

\author{Kazuyoshi Tatsumi} \email{k-tatsumi@imass.nagoya-u.ac.jp}
\affiliation{Advanced Measurement Technology Center, Institute of
Materials and Systems for Sustainability, Nagoya University, Chikusa,
Nagoya 464-8603, Japan}
\affiliation{Center for Elements Strategy
Initiative for Structural Materials, Kyoto University, Sakyo, Kyoto
606-8501, Japan}

\author{Atsushi Togo}
\affiliation{Center for Elements Strategy Initiative for Structural
Materials, Kyoto University, Sakyo, Kyoto 606-8501, Japan}

\author{Isao Tanaka}
\affiliation{Center for Elements Strategy Initiative for Structural
Materials, Kyoto University, Sakyo, Kyoto 606-8501, Japan}
\affiliation{Department of Materials Science and
Engineering, Kyoto University, Sakyo, Kyoto 606-8501, Japan}
\affiliation{Nanostructures Research Laboratory, Japan Fine Ceramics
Center, Atsuta, Nagoya 456-8587, Japan}

\begin{abstract}
Lattice thermal conductivities of $\alpha$-, $\beta$- and $\gamma$-Si$_3$N$_4$
single crystals are investigated from {\it ab-initio} anharmonic lattice
dynamics, within the single-mode relaxation-time approximation of the
linearized phonon Boltzmann transport equation. At 300 K, the lattice thermal
conductivity of $\beta$-Si$_3$N$_4$ is calculated as $\kappa_{xx}=73$ and
$\kappa_{zz}=198$ (in units of Wm$^{-1}$K$^{-1}$), that is consistent with the
reported experimental values of 69 and 180, respectively. For
$\alpha$-Si$_3$N$_4$, $\kappa_{xx}=69$ and $\kappa_{zz}=99$ are obtained.  The
difference of anisotropy between $\alpha$-Si$_3$N$_4$ and $\beta$-Si$_3$N$_4$
is originated from their characteristic difference in the phonon band
structures, although their crystal structures are similar in their local atomic
coordinates. In $\alpha$-Si$_3$N$_4$, acoustic-mode phonons below 6 THz are the
main heat carriers. In $\beta$-Si$_3$N$_4$, the phonon modes up to 12 THz
contribute to the lattice thermal conductivity. In $\gamma$-Si$_3$N$_4$,
$\kappa=81$ is obtained. The distribution of phonon mode contributions to
lattice thermal conductivity with respect to phonon frequency is found to
closely resemble $\kappa_{xx}$ of $\beta$-Si$_3$N$_4$ although the phonon
lifetimes of $\gamma$-Si$_3$N$_4$ are twice shorter than those of
$\beta$-Si$_3$N$_4$.
\end{abstract}

\maketitle

\section{Introduction}
Several nitride insulators are known to exhibit high thermal conductivities and
are important for heat transfer materials at elevated temperatures. For
example, Slack {\it et al.}~\cite{slack} reported that wurtzite-type w-AlN has
thermal conductivity of $\gg$  100 Wm$^{-1}$K$^{-1}$. \textcolor{blue}{Si$_3$N$_4$ has become
another promising thermal conductive insulator as its thermal conductivity has
been improved up to 177 Wm$^{-1}$K$^{-1}$ by using the advanced ceramic
technologies related to the densification and microstructure
control.~\cite{zhou,hirao-rev,watari,hirosaki}} Since the Si$_3$N$_4$ ceramics also
exhibit high mechanical strength at elevated temperatures, they are
regarded as ideal materials for the use in various applications, such as
engine components, gas turbines, and heat sink substrates of power
semiconductor devices.

At atmospheric pressure, Si$_3$N$_4$ exists in one of two phases, $\alpha$ and
$\beta$, which are generally considered as low- and high-temperature phases,
respectively.~\cite{zhou,hirosaki,riley} \textcolor {blue}{Their crystal structures belong to the
	space groups of P31c and P6$_3$/m, respectively.~\cite{yashima,boulay} These structures have 
different stacking orders of equivalent basal layer structures originated by SiN$_4$
tetrahedra.~\cite{hampshire} In Fig.\ref{fig:Fig1_cryst}  these layer structures are depicted from
the principal direction, as A, B, C, and D in the $\alpha$ phase and A and B in
the $\beta$ phase. The stacking manners in $\alpha$- and $\beta$-Si$_3$N$_4$ are
thus as ABCDABCD.. and ABAB.., respectively. 
The $\alpha$ phase has additional two layer structures of C and D, which are related to the A and B by
the $c$ glide operation.~\cite{hampshire} Along this direction the lattice constant of the $\alpha$ phase is
approximately two times longer than that of the $\beta$ phase.}
\begin{figure}[ht]
 \begin{center}
  \includegraphics[width=0.90\linewidth]{Fig1_crystal_str2.eps} \caption{(color
  online) Crystal structures of $\alpha$- and $\beta$-Si$_3$N$_4$. Stacking of
  SiN$_4$ tetrahedron layers are shown in left. (a) ABCDABCD.. for
  $\alpha$-Si$_3$N$_4$. (b) ABAB.. for $\beta$-Si$_3$N$_4$.  Space group
  diagrams~\cite{inttableA} in P31c ($\alpha$-Si$_3$N$_4$) and P6$_3$/m ($\beta$-Si$_3$N$_4$)
  are shown in right.}
  \label{fig:Fig1_cryst} 
 \end{center}
\end{figure}

\textcolor{red}{The reported values~\cite{zhou,hirao-rev,watari,hirosaki,hirai}
of the thermal conductivity of the Si$_3$N$_4$ polymophs were measured on
pollycrystalline bulk samples. These values were significantly affected by
the contained lattice defects, impurities, shapes and orientations of the
constituent crystal grains;~\cite{hirosaki-md} the thermal conductivity intrinsic to a
defect-free single crystal has not been established. As
an experimental access to the intrinsic thermal conductivity,} Li {\it et
al.}~\cite{li} applied high-resolution thermoreflectance microscopy to a single
$\beta$-Si$_3$N$_4$ grain in a ceramic sample. The analyzed thermal
conductivities were 69 and 180 Wm$^{-1}$K$^{-1}$ along the $a$ and $c$ axes,
respectively. \textcolor{red}{These thermal quantities correspond to the diagonal elements
$\kappa_{xx}$ and $\kappa_{zz}$ of the lattice thermal conductivity tensor
$\boldsymbol{\kappa}$.} We consider the anisotropy of
$\kappa_{zz}/\kappa_{xx}\sim 3$ is relatively large.  Theoretically, Hirosaki
{\it et al.}~\cite{hirosaki-md} estimated $\boldsymbol{\kappa}$ by
applying the Green-Kubo formulation to the molecular dynamics (MD) method with
the interatomic potentials proposed by Vashishta {\it et al.}~\cite{vashishta}. 
They calculated $\kappa$$_{xx}$ and $\kappa$$_{zz}$ of
$\alpha$-Si$_3$N$_4$ as 105 and 225 Wm$^{-1}$K$^{-1}$, and those of
$\beta$-Si$_3$N$_4$ as 170 and 450 Wm$^{-1}$K$^{-1}$, respectively.
The ratio $\kappa_{zz}/\kappa_{xx}$ in $\beta$-Si$_3$N$_4$ agreed well with the
experimental results obtained by Li {\it et al.}~\cite{li}: The $\kappa_{ii}$
values overestimated the experimental by more than two times. 

On many polymorphs of the wurtzite and zincblende structures, the lattice
thermal conductivities were recently calculated based on a first principles
calculation and Boltzman transport theory.~\cite{phono3py}  These polymorphs
have different stacking orders of the densest atom planes as ABAB.. and
ABCABC...  The different stacking orders merely altered the lattice thermal
conductivities.~\cite{phono3py} The phonon linewidth distribution and phonon
density of states were similar between the structures as well.~\cite{phono3py}
On the other hand, the previous MD results on $\alpha$- and
$\beta$-Si$_3$N$_4$ presented that the different stacking orders in these
phases altered the $\boldsymbol{\kappa}$ values largely. The large
difference has not been clarified based on the microscopic phonon properties.
It is interesting to investigate it using the same approach as in
Ref.\onlinecite{phono3py} for $\alpha$- and $\beta$-Si$_3$N$_4$.

In addition to the $\alpha$ and $\beta$ phases, a cubic spinel phase
($\gamma$-Si$_3$N$_4$) is known to form upon compression and in-situ
heating.~\cite{zerr,zhang} The reported transition pressures were scattered
from 10 to 36 GPa depending on the experimental conditions.~\cite{xu}  The
$\gamma$ phase is experimentally quenched to atmospheric pressure at room
temperature.  Its thermal conductivity has not been experimentally reported; it
has been estimated only by the Slack model.~\cite{morelli} 

The present study aims to qualitatively understand the lattice thermal
conductivity tensors among the three Si$_3$N$_4$ phases by means of the first
principles approach.  \textcolor{blue}{We calculate $\boldsymbol{\kappa}$ of
the $\gamma$ phase as well, for systematic understanding of the lattice
thermal conductivity in the Si$_3$N$_4$ system.  After the methodology
section, we examine the validity of the present results first.  Our
calculated thermal properties are compared with the available experimental
and theoretical references.  Then we investigate the characteristics in the
calculated $\boldsymbol{\kappa}$ in detail on the basis of the phonon band
structures and phonon linewidths.}

\section{Computational procedures}

\subsection{Lattice thermal conductivity calculation}

The lattice thermal conductivities were calculated by solving the linearized
Boltzmann transport equation (LBTE) within the single-mode relaxation time
approximation (single-mode RTA).  We also tried the direct-solution of
LBTE~\cite{chaput-direct} and leave its calculated $\boldsymbol{\kappa}$ values
in the following section. The difference in the calculated
$\boldsymbol{\kappa}$ between by the single-mode RTA and by the direct solution
was found minor for our discussion. Therefore we limited our research to use
the single-mode RTA to take advantage of its intuitive closed form of
$\boldsymbol{\kappa}$.

In the following sections, we denote a phonon mode by $\lambda=(\mathbf{q},p)$
with the set of the phonon wave vector $\mathbf{q}$ and band index $p$ and $-\lambda \equiv (-\mathbf{q},p)$. The
relaxation time due to phonon-phonon scattering was obtained as reciprocal of
linewidth, $\tau_{\lambda,\text{ph-ph}}=(2\Gamma_\lambda)^{-1}$, where the
linewidth that we employed in this study is as follows:
\begin{align}
 \label{eq:linewidth}
 &\Gamma_\lambda = \frac{18\pi}{\hbar^2}
  \sum_{\lambda' \lambda''}
  \bigl|\Phi_{-\lambda\lambda'\lambda''}\bigl|^2 \times \nonumber \\ 
 &\left\{ (n_{\lambda'} + n_{\lambda''}+1) 
   \delta(\omega_\lambda-\omega_{\lambda'}-\omega_{\lambda''}) \right.
   + \nonumber \\ 
 &\;\;(n_{\lambda'}-n_{\lambda''})
  \left[\delta(\omega_\lambda +\omega_{\lambda'}-\omega_{\lambda''})
 \right. 
 \left. -\left. \delta(\omega_\lambda - \omega_{\lambda'}+\omega_{\lambda''})
 \right]\right\}.
\end{align}
Here $\omega_\lambda$ is the harmonic phonon frequency of the phonon mode
$\lambda$, $n_\lambda=[\exp(\hbar\omega_\lambda/\mathrm{k_B}T)-1]^{-1}$ is
the Bose-Einstein distribution at temperature $T$, and
$\Phi_{\lambda\lambda'\lambda''}$ denotes the three-phonon-scattering strength.
$\Phi_{\lambda\lambda'\lambda''}$ was obtained by usual coordinate
transformation of third-order force constants from direct space to phonon
space.~\cite{phono3py} The second- and third-order real-space force constants
were obtained from the {\it ab-initio} calculation, whose details are written in the
next section.

In order to compare the more realistic results of the calculated $\boldsymbol{\kappa}$ with the
experimental data, the isotopic scattering effect due to the natural isotope
distribution was taken into account according to the second-order perturbation
theory.~\cite{tamura} With the relaxation times of the phonon-phonon scattering
and isotopic scattering, $\tau_{\lambda,\text{ph-ph}}$ and
$\tau_{\lambda,\text{iso}}$, the total relaxation time for a phonon mode was
assumed to be $1/\tau_{\lambda} = 1/\tau_{\lambda,\text{ph-ph}} +
1/\tau_{\lambda,\text{iso}}$, according to Matthiessen's rule.

The available experimental thermal conductivity data of the Si$_3$N$_4$ system
have been measured on the polycrystalline samples and not measured from any
single crystals. In order to consider the effect of various lattice defects in
the polycrystalline samples, such as grain boundaries, impurities, and
vacancies, we crudely took them into account by a relaxation time
$\tau_{\lambda,\text{bs}}=L/|\mathbf{v}_\lambda|$ of a phonon boundary
scattering model, where $\mathbf{v}_\lambda =
\nabla_{\mathbf{q}}\omega_\lambda$ is the group velocity and $L$ a
parameter regarding to the boundary mean free path. We consider
$\tau_{\lambda,\text{bs}}$ as a variable parameter and included it to
$\boldsymbol{\kappa}$ according to Matthiessen's rule.

The closed form of $\boldsymbol{\kappa}$ within RTA was obtained via
\begin{align}
 \label{eq:kappa}
 \boldsymbol{\kappa}(T) = \frac{1}{N_\mathbf{q}\Omega} \sum_\lambda
 \tau_\lambda(T) \mathbf{v}_\lambda \otimes \mathbf{v}_\lambda c_\lambda(T),
\end{align}
where $N_\mathbf{q}$ is the number of
$\mathbf{q}$-points, $\Omega$ is the unit cell volume, and $c_\lambda$
is the mode heat capacity. To analyze $\boldsymbol{\kappa}$ in detail, we calculate
the cumulative thermal conductivity:
\begin{align}
 \label{eq:cum-kappa}
 \boldsymbol{\kappa}^\text{c}(\omega) = \frac{1}{N_\mathbf{q}\Omega}
 \int_0^\omega \sum_\lambda
 \tau_\lambda(T) \mathbf{v}_\lambda \otimes \mathbf{v}_\lambda
 c_\lambda(T) \delta(\omega'-\omega)d\omega',
\end{align}
and its derivative $\frac{\partial
\boldsymbol{\kappa}^\text{c}(\omega)}{\partial \omega}$ to see the phonon mode
contributions to $\boldsymbol{\kappa}$.

The lattice thermal conductivities were calculated with the phonon-phonon interaction calculation code
PHONO3PY~\cite{phono3py}, while the harmonic phonon states were analyzed with
the phonon calculation code PHONOPY~\cite{phonopy}.

\subsection{Computational details}

The force constants were calculated using the first-principles projector
augmented wave method~\cite{paw} (VASP code~\cite{vasp-1996,vasp-1995,
vasp-1999}). The generalized gradient approximation (GGA) parameterized by
Perdew, Burke, and Ernzerhof~\cite{pbe} was used for the exchange correlation
potential. A plane wave energy cutoff of 500 eV was employed. The crystal
structures were optimized until the residual forces acting
on the constituent atoms was less than $10^{-6}$ eV/\AA. The structural
optimization was firstly performed for a temperature of 0 K and 0 GPa. Here the
temperature and pressure were considered only for the electronic system and the
zero point lattice vibration was not taken into account. The calculated lattice
parameters were $a=7.808$ \AA~ and $c=5.659$ \AA~ for the $\alpha$ phase,
$a=7.660$ \AA~ and $c=2.925$ \AA~ for the $\beta$ phase, and $a=7.787$ \AA~
for the $\gamma$ phase, which agree with the experimental
data~\cite{yashima,boulay,paszkowicz} within +0.7 \% errors. The lattice
volume optimized within the local density approximation
(LDA)~\cite{lda} for the exchange correlation potential was, for
$\beta$-Si$_3$N$_4$, 3 \% smaller than within GGA, which is a typical
volume contraction of LDA. The $\boldsymbol{\kappa}$ calculated within LDA was larger by 2.6 \% than within GGA. For
our discussion, this difference is enough small, therefore the impact of
choice of exchange correlation potential is considered to be minor in our
study.

\begin{table}[ht]
	\caption{\label{table:LTC} Calculated lattice thermal conductivities 
 of $\alpha$-, $\beta$-, and $\gamma$-Si$_3$N$_4$
 (WK$^{-1}$m$^{-1}$) at 300 K with respect to several combinations of
 supercell sizes.}
 \begin{ruledtabular}
  \begin{tabular}{ccccc}
   \multirow{2}{*}{Phase}
   & \multicolumn{2}{c}{Supercell (\# of atoms)} &
   \multicolumn{2}{c}{LTC} \\
   \cline{2-5}
   & $3^\text{rd}$ force constants & $2^\text{nd}$ force constants & $xx$ & $zz$ \\
   \hline
   \multirow{6}{*}{$\alpha$}
   & $1\times 1\times 1$ (28) & $1\times
   1\times 1$ (28) & \textcolor{blue}{37} &   \textcolor{blue}{57} \\ 
   & $1\times 1\times 2$ (56) & $1\times
   1\times 2$ (56) & \textcolor{blue}{41} &   \textcolor{blue}{79} \\ 
   & $1\times 1\times 1$ (28) & $2\times
   2\times 2$ (224) & \textcolor{blue}{55} &   \textcolor{blue}{81} \\ 
   & $1\times 1\times 2$ (56) & $2\times
   2\times 2$ (224) & \textcolor{blue}{67} &   \textcolor{blue}{95} \\ 
   & $1\times 1\times 2$ (56) & $2\times
   2\times 3$ (336) & \textcolor{blue}{68} &  \textcolor{blue}{97} \\ 
   & $1\times 1\times 2$ (56) & $3\times
   3\times 4$ (1008) & \textcolor{blue}{68} &  \textcolor{blue}{100} \\ 
   \hline
   \multirow{5}{*}{$\beta$}
   & $1\times 1\times 2$ (28) & $1\times
   1\times 2$ (28) & \textcolor{blue}{44} & \textcolor{blue}{173} \\ 
   & $1\times 1\times 2$ (28) & $2\times
   2\times 4$ (224) & \textcolor{blue}{76} &  \textcolor{blue}{208} \\ 
   & $1\times 1\times 3$ (42) & $2\times
   2\times 4$ (224) & \textcolor{blue}{71} & \textcolor{blue}{194} \\ 
   & $1\times 1\times 3$ (42) & $2\times
   2\times 5$ (280) & \textcolor{blue}{72} & \textcolor{blue}{196} \\ 
   & $1\times 1\times 3$ (42) & $3\times
   3\times 8$ (1008) & \textcolor{blue}{73} & \textcolor{blue}{199} \\ 
   \hline
   \multirow{3}{*}{$\gamma$}
   & $1\times 1\times 1$ (56) & $1\times
   1\times 1$ (56) & \multicolumn{2}{c}{\textcolor{blue}{72}} \\ 
   & $1\times 1\times 1$ (56) & $2\times
   2\times 2$ (448) & \multicolumn{2}{c}{\textcolor{blue}{77}} \\ 
   & $1\times 1\times 1$ (56) & $3\times
   3\times 3$ (56) & \multicolumn{2}{c}{\textcolor{blue}{79}} \\ 
  \end{tabular}
 \end{ruledtabular}
\end{table}

Supercell and finite difference approaches were used to calculate the
force constants.~\cite{wei-supercell} The supercells were $1\times 1\times2$, $1\times
1\times3$, and $1\times 1\times1$ supercells of the conventional unit cells for
the calculation of the third-order force constants in $\alpha$, $\beta$, and
$\gamma$-Si$_3$N$_4$, respectively; they were $3\times 3\times4$, $3\times
3\times8$ and $2\times 2\times2$ for the second-order force constants.  The length of the
induced atomic displacement was set to 0.03 \AA.  Table \ref{table:LTC} shows
the $\boldsymbol{\kappa}$  calculated with several different sets of the supercells,
indicating that our calculated $\boldsymbol{\kappa}$ is reasonably converging with respect to
the size of the supercells. 

\textcolor{blue}{Non-analytical term correction~\cite{wang} was applied to the
second-order force  constants to take into account the long range Coulomb forces present in
ionic crystals. For the correction, Born effective charges were calculated by
using the density functional perturbation theory (DFPT).}\textcolor{red}{Ref!}

Uniform $\mathbf{k}$-point sampling meshes of $4\times 4\times 2$,
$4\times 4\times 3$, and $3\times 3\times 3$ were used for the
third-order force constants of the $\alpha$, $\beta$, and $\gamma$ phases. For the
$\alpha$ and $\beta$ phases the center of the $a^*$-$b^*$ plane was sampled
while the center on the $c^*$-axis was not. For the
$\gamma$ phase, non-$\Gamma$ center mesh was used. For the second-order
force constants, the $\Gamma$-point was only sampled for the $\alpha$ and $\beta$
phase and the only one $\mathbf{k}=(0.5, 0.5, 0.5)$ point was
sampled for the $\gamma$ phase. The $\mathbf{q}$-point
sampling meshes of $10\times 10\times 14$, $10\times 10\times 26$, and
$12\times 12\times 12$ were used to calculate $\boldsymbol{\kappa}$ in Eq.~(\ref{eq:kappa})
for the $\alpha$, $\beta$, and $\gamma$ phases.

\textcolor {blue}{We examined the effect of thermal expansion on
$\boldsymbol{\kappa}$ by calculating them with the crystal structures
optimized for several finite temperatures within the quasi-harmonic
approximation (QHA)~\cite{dove-p76}.  The structure was optimized for a
temperature and 0 GPa so as to show a minimum Helmholtz free energy.
$\boldsymbol{\kappa}$ for the temperature was calculated with the optimized
structure.  This $\boldsymbol{\kappa}$ was differed from the one for the
temperature, calculated with the structure optimized for 0 K and 0 GPa. We
consider that this difference is caused by the thermal expansion of the
lattice. For $\beta$-Si$_3$N$_4$ and the temperatures of 300, 600, 900,
1200, and 1500 K, the differences in $\boldsymbol{\kappa}$ between the two
structures were found less than 1 \%, similar to the case of Si and
Ge~\cite{ward-ltc}. For the present study, these differences are negligible
and we adopt the $\boldsymbol{\kappa}$ calculated with the structure
optimized for 0 K and 0 GPa.}


We calculated the volumetric thermal expansion coefficients and compared them
with the reported experimental values so as to check the validity of the
present thermal conductivity calculation, because the thermal expansion is
originated from the anharmonicity of the interatomic potential as well as
$\boldsymbol{\kappa}$. The calculated values of the $\alpha$ and $\beta$ phases
are 4.31$\times 10^{-6}$ and 4.19$\times 10^{-6}$ K$^{-1}$ for 300 K , while
the experimental values~\cite{minikayev-alpha} were 3.75$\times 10^{-6}$ and
3.55$\times 10^{-6}$ K$^{-1}$). The present calculation reproduced the
experimental tendency where the $\alpha$ phase has a slightly larger thermal
expansion coefficient than the $\beta$ phase, supporting that the present
calculations enable us to qualitatively compare the calculated
$\boldsymbol{\kappa}$ of the Si$_3$N$_4$ phases.

In order to compare the microscopic phonon properties among the three phases at
the same conditions, those results calculated at 0 GPa are shown and discussed.
For the $\gamma$ phase, this means that we assume the condition of a virtually
quenched $\gamma$ phase at 0 GPa from the high pressure. To examine the
analytical continuity of the properties with respect to pressures, we
calculated $\boldsymbol{\kappa}$ of the $\gamma$ phase at 10, 20, and 40 GPa as
shown in Fig.~\ref{fig:S1}. The phenomenological behaviour of linear dependence
of $\boldsymbol{\kappa}$ with respect to pressure was reproduced as similar to
Ref.~\onlinecite{andersson-pressure}. The slope was 2.89
Wm$^{-1}$K$^{-1}$GPa$^{-1}$ for the $\gamma$ phase.  By this dependence, we
consider that the microscopic values are also varied smoothly with the pressure
and those at 0 GPa are valuable to compare with the $\alpha$ and $\beta$
phases.

\subsection{Direct solution of LBTE}

The merit to employ the single-mode RTA for thermal conductivity calculation is
the closed form, by which we can intuitively understand the qualitative
character of $\boldsymbol{\kappa}$ in terms of the relaxation time and group velocity. The
microscopic understanding of the full solution of LBTE is still under the
development~\cite{cepellotti-relaxons} and the microscopic picture based on
collective phonons~\cite{hardy-collective} will require more complicated
investigation although it is known that the single-mode RTA solution of LBTE
often underestimates the full solution.~\cite{mukhopadhyay-ltc,ward-ltc}

On the $\boldsymbol{\kappa}$ of the $\alpha$ and $\beta$ phases, we adopted a
direct solution of LBTE~\cite{chaput-direct}, which is one of the methods of
LBTE full solutions. Their $\kappa_{xx}$ and $\kappa_{zz}$ without the isotope
effect were 69 and 102 Wm$^{-1}$K$^{-1}$ for  the $\alpha$ phase and 76 and 238
Wm$^{-1}$K$^{-1}$ for the $\beta$ phase, respectively, while the corresponding
single-mode RTA values were 70 and 102 Wm$^{-1}$K$^{-1}$ for the $\alpha$ phase
and 76 and 210 Wm$^{-1}$K$^{-1}$ for the $\beta$ phase. The $\kappa_{zz}$ of
the direct solution in the $\beta$ phase was 13 \% larger than that of the
single-mode RTA solution. Since the differences in $\boldsymbol{\kappa}$
between the LBTE solutions are not significant, we expect the physics on those
lattice thermal conductivities is well understood within RTA in the current
level of our interest.  \textcolor {red} {Therefore, we discuss the lattice
thermal conductivities using  the results of the single-mode RTA solution.}

\section{Results and discussion}

\subsection{Lattice thermal conductivities}

\begin{table}[ht]
 \caption{\label{table:LTC-exp} Calculated thermal conductivities of
 $\alpha$-Si$_3$N$_4$ (trigonal), $\beta$-Si$_3$N$_4$ (trigonal), and
 $\gamma$-Si$_3$N$_4$ (cubic) at 300
 K, compared with the experimental data. Theoretical bulk moduli $B$ in
 units of GPa, calculated by the authors by using the present band
 method, are presented in the fourth column.}
% \begin{ruledtabular}
%  \begin{tabular}{cccccccccc}
%   & \multicolumn{3}{c}{This work} & \multicolumn{3}{c}{\textcolor{red}{Ref. Theo.}}
%   & \multicolumn{3}{c}{Ref. Expt.} \\
%   \cline{2-10}
%   & $\kappa_{xx}$ & $\kappa_{zz}$ & $B$ & $\kappa$ & $\kappa_{xx}$ & $\kappa_{zz}$ & $\kappa$ & $\kappa_{xx}$ & $\kappa_{zz}$ \\
%   \hline
%   $\alpha$-Si$_3$N$_4$ & \textcolor{blue}{68} & \textcolor{blue}{100} & 224 & 70\footnotemark[1] & 105\footnotemark[2] & 225\footnotemark[2] & 59\footnotemark[4] & - & -  \\
%   $\beta$-Si$_3$N$_4$ & \textcolor{blue}{73} & \textcolor{blue}{199} & 237 & 250\footnotemark[1] & 170\footnotemark[2] & 450\footnotemark[2] & 122\footnotemark[5] & 69\footnotemark[6] & 180\footnotemark[6] \\
%   $\gamma$-Si$_3$N$_4$ & \textcolor{blue}{77} & - & 296 & 80\footnotemark[1] & - & - & - & - & - 
%   \footnotetext[1]{Ref.~\onlinecite{morelli}, Slack model.}
%   \footnotetext[2]{Ref.~\onlinecite{hirosaki-md}, molecular dynamics (Green-Kubo).}
%   \footnotetext[4]{Ref.~\onlinecite{hirai}, thin film.}
%   \footnotetext[5]{Ref.~\onlinecite{hirosaki}, poly-crystals.}
%   \footnotetext[6]{Ref.~\onlinecite{li}, single crystalline grains of poly-crystals.}
%  \end{tabular}
% \end{ruledtabular}
%\end{table}

\begin{ruledtabular}
 \begin{tabular}{ccccccccc}
   & \multicolumn{3}{c}{This work} & \multicolumn{3}{c}{\textcolor{red}{Ref. Theo.}}
   & \multicolumn{2}{c}{Ref. Expt.} \\
   \cline{2-9}
   & $\kappa_{xx}$ & $\kappa_{zz}$ & $B$ & $\kappa$ & $\kappa_{xx}$ & $\kappa_{zz}$ & $\kappa_{xx}$ & $\kappa_{zz}$ \\
   \hline
   $\alpha$-Si$_3$N$_4$ & \textcolor{blue}{68} & \textcolor{blue}{100} & 224 & 70\footnotemark[1] & 105\footnotemark[2] & 225\footnotemark[2] & - & -  \\
   $\beta$-Si$_3$N$_4$ & \textcolor{blue}{73} & \textcolor{blue}{199} & 237 & 250\footnotemark[1] & 170\footnotemark[2] & 450\footnotemark[2] & 69\footnotemark[3] & 180\footnotemark[3] \\
   $\gamma$-Si$_3$N$_4$ & \textcolor{blue}{77} & - & 296 & 80\footnotemark[1] & - & - & - & - 
   \footnotetext[1]{Ref.~\onlinecite{morelli}, Slack model.}
   \footnotetext[2]{Ref.~\onlinecite{hirosaki-md}, molecular dynamics (Green-Kubo).}
   \footnotetext[3]{Ref.~\onlinecite{li}, single crystalline grains of poly-crystals.}
  \end{tabular}
 \end{ruledtabular}
\end{table}

Table \ref{table:LTC-exp} shows the present results of the
$\boldsymbol{\kappa}$ for 300 K.  $\beta$-Si$_3$N$_4$ has markedly more
anisotropic $\boldsymbol{\kappa}$ than $\alpha$-Si$_3$N$_4$.  The directional
averages $\sum_i \kappa_{ii}/3$  are 79, 115,  and 77 Wm$^{-1}$K$^{-1}$ for the
$\alpha$, $\beta$, and $\gamma$ phases, respectively.  The value of the
$\gamma$ phase is similar to that of the $\alpha$ phase, in spite of
comparatively large difference among the bulk moduli ($B$) that are also shown
in Table \ref{table:LTC-exp}.   

Table \ref{table:LTC-exp} also shows the previously reported
experimental~\cite{li} and theoretical~\cite{hirosaki-md} thermal
conductivities for the references.  Previously Morelli {\it et
al.}~\cite{morelli} employed the Slack model for estimating the lattice thermal
conductivities of the three phases. They are shown as $\kappa$ in Table
\ref{table:LTC-exp}. For the $\beta$ phase, our $\boldsymbol{\kappa}$ agrees
better with the experimental, than that of the molecular
calculation~\cite{hirosaki-md} does.  Also, our directional average $\sum_i
\kappa_{ii}/3$ (115 Wm$^{-1}$K$^{-1}$) agrees better with the experimental
average (106 Wm$^{-1}$K$^{-1}$), than the Slack model does.

Fig.~\ref{fig:Fig1_338} shows the theoretical $\boldsymbol{\kappa}$ of the
$\alpha$ and $\beta$ phases as a function of $T$, together with the reference
experimental data~\cite{hirosaki,hirai,li}. The thermal conductivities for a
series of temperatures were only reported on the polycrystalline bulk samples.
These bulk thermal conductivities cannot be directly compared with the
calculated intrinsic $\boldsymbol{\kappa}$ because the microstructures of the
bulk samples largely affected the bulk thermal conductivities:  \textcolor
{blue} { They were deviated from the simple directional averages of the
intrinsic $\kappa_{ii}$, depending on the shapes of the constituent crystal
grains.  We treated this effect by using a parameter $0\le{w}\le{1}$ and
fitting quantities of $w\kappa_{xx} + (1-w) \kappa_{zz}$ to the experimental
bulk thermal conductivities by the least squares method. We consider these as
theoretical bulk thermal conductivities }

\begin{figure}[ht]
 \begin{center}
  \includegraphics[width=0.90\linewidth]{Fig1_m1010.eps} \caption{(color
  online) Temperature dependence of thermal conductivities for $\alpha$- and
  $\beta$-Si$_3$N$_4$. For $\beta$-Si$_3$N$_4$, theoretical results with the
  boundary scattering effect are shown by broken lines. Theoretical bulk
  thermal conductivities (see in text) for the $\beta$-Si$_3$N$_4$ sample are
  also shown to be compared with the experimental bulk thermal conductivities.}
  \label{fig:Fig1_338}
 \end{center}
\end{figure}

\textcolor {blue}{In Fig.~\ref{fig:Fig1_338}, the $\kappa_{ii}$ calculated
without $\tau_{\lambda,\text{bs}}$ are nearly proportional to $T^{-1}$ because
$n_\lambda$ in Eq.~(\ref{eq:linewidth}) can be reduced to
$\exp(-\hbar\omega_\lambda/\mathrm{k_B}T)$.} In Fig.~\ref{fig:Fig1_338}-a, the
temperature dependence of the experimental bulk thermal conductivities of a
chemically vapor-deposited $\alpha$-Si$_3$N$_4$ sample~\cite{hirai} is deviated
from inverse proportion considerably, intersecting the theoretical lines.  Thus
no value of $w$ adjusts the theoretical conductivities to the experimental
data.  The full solution of LBTE would negligibly cure the disagreement.
Including the simple phonon lifetime of boundary scattering,
$\tau_{\lambda,\text{bs}}=L/|\mathbf{v}_\lambda|$, into the total phonon
lifetime according to Matthiessen's rule, could not explain the discrepancy as
well.  \textcolor {blue}{A $L$ value of 0.6 $\mu\text{m}$, which was much
smaller than the experimental grain size~\cite{hirai} of 10 $\mu\text{m}$,} decreased the
room-temperature theoretical $\kappa$$_{xx}$ and $\kappa$$_{zz}$ values toward
the experimental values, but severely underestimated the experimental values in
the high temperature side.  At present, the reason for the discrepancy between
the theoretical and experimental behaviors is unclear.  Although the crystal
structure of the experimental sample was characterized as $\alpha$-Si$_3$N$_4$,
significant lattice defects might exist in the as-deposited sample as pointed
out by Hirosaki {\it et al.}~\cite{hirosaki-md} and \textcolor{blue}{the simple
phonon boundary scattering model might fail to describe their effects on the
bulk thermal conductivities.} 

The experimental bulk thermal conductivities of the $\beta$ phase are located
in-between the theoretical $\kappa$$_{xx}$ and  $\kappa$$_{zz}$, being nearly
proportional to $T^{-1}$. Simple directional averages of the theoretical
$\kappa_{ii}$ slightly underestimate these experimental values.  This is
understood from the fact that the microstructure was controlled to improve the
thermal conductivity, and the crystalline grains were selectively grown along
the $c$ axis of the most conductive direction.~\cite{hirosaki} The theoretical
bulk thermal conductivities were fit well with $w=\textcolor{blue}{0.44}$  to
the experimental bulk thermal conductivities.  For the effects of lattice
defects most of which were grain boundaries, we included
$\tau_{\lambda,\text{bs}}$ with $L = \textcolor{blue}{0.6}$ $\mu\text{m}$ to
further fit the theoretical curve ($w=\textcolor{blue}{0.33}$) to the
experimental data.  The $L$ value is \textcolor{blue}{slightly smaller than}
the average grain size~\cite{hirosaki} (2 $\mu\text{m}$) of the experiment.
The experimental $\kappa$$_{ii}$ are rather close to the theoretical
$\kappa$$_{ii}$ calculated without $\tau_{\lambda,\text{bs}}$.  This is
explained by the fact that the experimental $\kappa$$_{xx}$ and
$\kappa$$_{zz}$ were deduced consistently through the grains of several
different sizes. The effects of the phonon scattering at the grain boundaries
were eliminated.~\cite{li}


\subsection{Dispersion curves}

Figure \ref{fig:Fig4_ver5_338} shows the phonon band diagrams of the three
Si$_3$N$_4$ phases. The entire band diagrams are almost identical to those
reported earlier~\cite{kuwabara,xu}. However, here we investigate the group
velocities projected on the high-symmetry paths, especially focusing on their
anisotropy of the $\alpha$ and $\beta$ phases. This was
not investigated by the previous works.

\begin{figure}[ht]
 \begin{center}
  \includegraphics[width=0.90\linewidth]{Fig4_ver5_338_resize2_woDOS.eps}
  \caption{(color online) Brillouin-zones (left) and calculated phonon band diagrams (right) for three Si$_3$N$_4$ phases.
  \label{fig:Fig4_ver5_338} }
 \end{center}
\end{figure}

%The acoustic branches in the $\alpha$ phase highlighted in red in
%Fig.~\ref{fig:Fig4_ver5_338}-a do not increase their frequencies much more than
%those along the other paths, $\Gamma$--K or $\Gamma$--M. The frequency maxima
%along the $\Gamma$--A path are around 7 THz, rather close to the maxima along
%the $\Gamma$--K and $\Gamma$--M paths (around 5 THz). The upper branches along
%the $\Gamma$--A path are also as flat as the upper branches along the
%$\Gamma$--K and $\Gamma$--M paths.  In contrast, in the band diagram of the
%$\beta$ phase (Fig.~\ref{fig:Fig4_ver5_338}-b), the acoustic phonon branches
%highlighted in red along the $\Gamma$--A path increase their frequencies almost
%linearly from the $\Gamma$-point to the A-point and reach around 10 THz, along
%which the group velocity component $v_{\lambda,z}$ maintains high values.
The $\omega_{\lambda}$ on the acoustic branches in the $\beta$ phase increase
much more from $\Gamma$ to A than from $\Gamma$ to K or M.  In the $\alpha$
phase, the corresponding $\omega_{\lambda}$ increase more equally among the
paths.  This difference is due to the different $\Gamma$-A path lengths.  The
$\beta$ phase has an approximately twice longer path than the $\alpha$ phase;
the lattice constant $c$ of the $\beta$ phase is nearly half that of the
$\alpha$ phase, owing to the difference in the stacking manner of the basal
layers. Normally, optical branches are flat; however, the upper branches along
the $\Gamma$--A path also have significantly large gradients in the $\beta$
phase.  These findings indicate that the $\beta$ phase contains a larger number
of phonon modes whose ${v}_{\lambda,z}$ is large and whose $\mathbf{v}_\lambda$
orients to the $c$ axis direction. We will show this tendency further in the
following sections.

In the $\gamma$ phase, the acoustic phonon branches show significant linear
dispersions on the $\Gamma$--L and $\Gamma$--X paths.  The roughly constant
gradients of $\omega_{\lambda}$ are large, reflecting the large elastic
constants of the $\gamma$ phase as shown in Table \ref{table:LTC-exp}.

\subsection{$\omega_\lambda$ counter map on reciprocal plane}

\begin{figure}[ht]
 \centerins
  \includegraphics[width=\linewidth]{Fig2_small.eps} \caption{(color
  online) Contour maps of phonon frequency (THz) on the $b^*$-$c^*$
  planes of Brillouin-zones. The coordination in the reciprocal plane 
   are in units of $10^{-2}$ \AA$^{-1}$. The maps for the four lowest-frequency
  phonon states are shown. The frequency landscapes are formed by simply
  connecting the frequencies of the same band indices, assigned by
  ascending order of frequency at the respective $\mathbf {q}$
  points. \label{fig:Fig3_338} }
 \centering
\end{figure}

We show the anisotropy in the group velocities on another geometry, that is,  a
cross-section of the Brillouin-zone.  In Fig.~\ref{fig:Fig3_338}, we show the
distributions on the $b^*$-$c^*$ plane.  The distributions for four bands from
the lowest frequency are shown, because these bands contribute significantly to
the $\boldsymbol{\kappa}$.  We did not find any significant differences in the
distributions between the $b^*$-$c^*$ plane and the other planes containing the
$c^*$ axis.  Thus we select the $b^*$-$c^*$ plane as a representative plane.  In
the $\alpha$ phase, the distributions are nearly isotropic. Their gradients,
the group velocities projected onto the $b^*$-$c^*$ plane, are thus nearly
isotropic. On the other hand, in the $\beta$ phase, the iso-frequency lines in
$0.06 \le q_{c^*} \le 0.12$ \AA$^{-1}$ are rather parallel to the $q_{b^*}$
axis, confirming that the four bands of the $\beta$ phase, in a significantly
large part of the Brillouin zone, have $\mathbf{v}_\lambda$ oriented to the $c$
axis direction.

\subsection{Frequency-dependences of $\boldsymbol{\kappa}^\text{c}$, $\mathbf{v}$$_\lambda$ and $\Gamma_\lambda$}

\begin{figure*}[ht]
 \begin{center}
  \includegraphics[width=0.9\linewidth]{figure_dos_jdos_kc_m1010.eps}
  \caption{(color online) Microscopic phonon properties of three Si$_3$N$_4$
	  phases. Cumulative thermal conductivity $\mathbf{\kappa}^\text{c}$ and its derivative
	  (a), DOS (b), weighted DOS with $v_{\lambda,i}^2$ (c) and linewidth $\Gamma_\lambda$ (d).
  \label{fig:Fig5_338_rev} }
 \end{center}
\end{figure*}

We have investigated in the previous two sections the anisotropy in the
$\mathbf{v}_\lambda$ for obtaining an insight on the anisotropy in the
$\boldsymbol{\kappa}$. Here we completely investigate the characteristic points
in the $\boldsymbol{\kappa}$ by using the important phonon properties existing
in the closed form of RTA in Eq.~(\ref{eq:kappa}). These properties are taken
over the Brillouin zone because the $\boldsymbol{\kappa}$ is a quantity
integrated there. In order to visually show the similarity or difference in
these properties among the phases, we plot in Fig.~\ref{fig:Fig5_338_rev} their
frequency distributions: The phonon density of states (DOS) shows the
distribution of the heat carriers. The cumulative thermal conductivities
$\boldsymbol{\kappa}^c$ show the phonon frequencies where the phonons largely
contribute to the $\boldsymbol{\kappa}$. Their first derivatives are also shown
to clearly identify these frequencies. By weighting the DOS with
$v_{\lambda,i}^2$, we show the impacts of the $v_{\lambda,i}$ and carrier
density together. We abbreviate these weighted DOS as WDOS. The phonon linewidth
is the remaining important property. To avoid from double counting of the carrier
density, scatter plots of $(\Gamma_\lambda, \omega_\lambda)$  are employed to
compare their impacts. 

The distinct first peaks in the DOS are indicated by arrows in
Fig.~\ref{fig:Fig5_338_rev}-a. They are roughly  related to the flattening of
the acoustic branches near the Brillouin zone boundaries. It is interesting that
only $\boldsymbol{\kappa}^c$ in the $\beta$ phase increase significantly in the
higher frequency range beyond the peak. This is due to the anisotropic dispersions of
the acoustic phonon branches and the large gradients of the low frequency
optical phonon branches.

Compared the four kinds of phonon properties among the three phases, the DOS,
WDOS and $\Gamma_\lambda$ distribution of the $\gamma$ phase are much different
from the other phases. These large differences are consistent with the large
differences in the crystal structure. Some characteristic points are explained
by the chemical bonding of the $\gamma$ phase. (1) In the DOS, the first peak is
located at a higher frequency than the other phases. This is consistent with the
finding in the band diagram; the linear dispersions of the acoustic phonon
branches with the large gradients, reflecting the strong chemical bonding
consistent with the large bulk modulus. (2) The WDOS is larger than the other
phases in the most part of the frequency range where
$\frac{d\kappa^c_{ii}}{d\omega_\lambda}$ is large. This is also explained by the
large gradients of the dispersions. (3) The linewidths are also larger than the
other phases in the frequency range. For this, the
three-phonon-scattering strength $\Phi_{\lambda\lambda'\lambda''}$ is shortly
investigated here.  In Table.~\ref{table:aveavepp}, the magnitudes of
$\Phi_{-\lambda\lambda'\lambda''}$ are compared as the averages over two
frequency ranges of $\omega_\lambda$ and all ($\lambda'$,$\lambda''$). The
values between 0--15 THz of the $\gamma$ phase are much larger than the other
phases. This is consistent with the larger linewidths. 
As a result of the {\it opposite} impacts of the linewidth and WDOS, the
$\kappa^c_{xx}$ resembles closely to $\kappa^c_{xx}$ of the $\beta$ phase.

\begin{table}[ht]
	\caption{\label{table:aveavepp} \textcolor{blue}{Averages of
	$\Phi_{-\lambda\lambda'\lambda''}$ over frequency ranges of
	$\omega_\lambda$ (0--15 and 0--35 THz) and all ($\lambda'$,$\lambda'$). The
	values are in units of 10$^{-10}$ eV$^2$f.u.$^{-1}$.}}
 \begin{ruledtabular}
  \begin{tabular}{cccc}
	  \multirow{2}{*}{Frequency Range (THz)}
   & \multicolumn{3}{c}{Phase}  \\
   \cline{2-4}
   & $\alpha$ & $\beta$ & $\gamma$ \\
   \hline
   \multirow{1}{*}{0--15}
   & \textcolor{blue}{2.66}  &  \textcolor{blue}{2.63}  & 5.76 &    
   \multirow{1}{*}{0--30}
   & \textcolor{blue}{13.1} & \textcolor{blue}{13.0} & 11.4 &     
  \end{tabular}
 \end{ruledtabular}
\end{table}

\begin{figure}[ht]
 \centering
  \includegraphics[width=0.9\linewidth]{figure_jdoss.eps} \caption{(color
	  online) \textcolor{blue}{JDOS of $\alpha$- and $\gamma$-Si$_3$N$_4$ at different $\mathbf q$ points.
  The first and forth rows are JDOS at the same $\Gamma$-point but calculated with the polarization for non-analytic term correction set along $c^*$ and $b^*$, respectively.} \label{fig:Fig6_338} }
 \centering
\end{figure}


We hereafter focus on the comarison between the $\alpha$ and $\beta$ phases. As
for the anisotropy in the $\boldsymbol{\kappa}$, we find clear similarity
between $\frac{d\kappa^c_{ii}}{d\omega_\lambda}$ and WDOS with
$v^2_{\lambda,i}$. The remaining property, linewidth distributions are similar
between the phases. The DOS are also similar. Therefore, the
$\mathbf{v}_\lambda$ simply account for the difference in the
$\boldsymbol{\kappa}$ anisotropy between the two phases.  

It is still curious that the linewidths are similar
between these phases although their group velocities have marked differences. 
Thus we investigate this similarity further.
Previously, Lindsay {\it et
al.}~\cite{Lindsay}
showed among many crystals of the zincblend structure a significant positive
correlation between the $\boldsymbol{\kappa}$ and phase space
available for the three-phonon scattering. The phase space was calculated by
counting the number of three phonon configurations satisfying the selection
rules for the energy and momentum conservation. The phase space is equivalent to
the integration of the joint density of states (JDOS) over the frequencies and
the Brillouin zone. 
More recently Togo {\it et al.} investigated
the imaginary part of self-energy
$\Gamma_{\lambda}(\omega)$.  The $\Gamma_{\lambda}$($\omega$) is
related to $\Gamma_{\lambda}$ by $\Gamma_{\lambda}$($\omega_{\lambda}$)=
$\Gamma_{\lambda}$. They showed that the frequency profile of
$\Gamma_{\lambda}$($\omega$) was characterized by JDOS.
~\cite{phono3py} Thus here we investigate the JDOS, ${D_2(\mathbf{q},\omega)}$,
\begin{align}
 \label{eq:jdos}
 &D_2(\mathbf{q},\omega) = D_2^{(1)}(\mathbf{q},\omega) +  D_2^{(2)}(\mathbf{q},\omega)
\end{align}
where 
\begin{eqnarray*}
	D_2^{(1)} & = & \frac{1}{N} \sum_{\lambda'\lambda''}\Delta(-\mathbf{q} + \mathbf{q'} + \mathbf{q''}) \nonumber \\
								   & \times & [\delta(\omega + \omega_{\lambda'} - \omega_{\lambda''}) + \delta(\omega - \omega_{\lambda'} + \omega_{\lambda''})],\\
	D_2^{(2)} & = & \frac{1}{N} \sum_{\lambda'\lambda''}\Delta(-\mathbf{q} + \mathbf{q'} + \mathbf{q''}) \nonumber \\
								   & \times & \delta(\omega - \omega_{\lambda'} - \omega_{\lambda''}),
\end{eqnarray*}
with $\Delta$($\mathbf{x}$) giving 1 if $\mathbf{x}$ is a reciprocal lattice vector and otherwise zero.
Fig.~\ref{fig:Fig6_338} shows the frequency-dependences of JDOS at different
$\mathbf{q}$-points on the $\Gamma$--A and $\Gamma$-K paths, which show very slight $\mathbf{q}$-point dependence.
Eq.~(\ref{eq:linewidth}) includes Bose-Einstein functions for the involved
phonon modes and JDOS can be weighted with them as done in
refs.~\onlinecite{mukhopadhyay-ltc,phono3py}, however we omit them for
simplicity. With the weights, the absolute values are affected but the weighted
JDOS of the $\alpha$ and $\beta$ phases are still similar. At the low frequency
region responsible for the LTCs, among the two terms of $D_2^{(1)}$ and
$D_2^{(2)}$ in Eq.~(\ref{eq:jdos}), dominant is $D_2^{(2)}$ which basically
corresponds to the half part ($\omega \geq  0$) of the auto-correlation
function of DOS, which, for the $\alpha$ and $\beta$ phases, commonly show the
frequency gap (Fig.~\ref{fig:Fig5_338_rev}-a).  $D_2^{(2)}$ curves reflect this
DOS feature, dropping suddenly around 0 THz and showing a small shoulder around
5 THz, which corresponds to the width of the gap. Moreover $D_2^{(2)}$ shows a
broad peak around 18 THz, which corresponds to the frequency shift to make the
largest correlation between the higher and lower portions of DOS across the
gap.  Because the gap is basically originated from the differences in the
vibrations of the planer NSi$_3$ commonly contained in the $\alpha$ and $\beta$
phases~\cite{kuwabara}, the major shapes of $D_2^{(2)}$, reflecting this gap
feature, are similar in these phases. With the same origin, the JDOS of
$D_2^{(1)}$ are also similar in these phases. 

Moreover, as indicated in Table.~\ref{table:aveavepp}, $\Phi_{-\lambda\lambda'\lambda''}$
have similar impacts on the linewidths. 
With these similar impacts of JDOS and $\Phi_{-\lambda\lambda'\lambda''}$ o n $\Gamma_\lambda$,
$\Gamma_\lambda$ in Fig.~\ref{fig:Fig5_338_rev}-d are similar.  

As a small but interesting difference in the linewidths between these phases,
$\Gamma_\lambda$ below 5 THz in Fig.~\ref{fig:Fig5_338_rev}-d are aligned on a
smooth line in the $\alpha$ phase, while those in the $\beta$ phase are
scattered roughly onto two different lines. This difference can be explained by
the vibration directions shown in Fig.~\ref{fig:Fig7_338}. In
Fig.~\ref{fig:Fig7_338}-a, $\Gamma_\lambda$ are classified using colors
according to the sums of the squares of the eigenvector components along $\mathbf{q}$; the
sum is 1 for perfectly longitudinal waves. However, these sums have no clear
contrast to distinguish the two branches in the $\beta$ phase.
Fig.~\ref{fig:Fig7_338}-b shows the same plot as Fig.~\ref{fig:Fig7_338}-a, but
with colors according to the sums of the squares of the eigenvector components
along the $a$-$b$ plane, which has 1 when the eigenvectors lie on the $a$-$b$
plane. There is a tendency in the $\beta$ phase that  $\Gamma_\lambda$ are
large for the vibrations along the $a$-$b$ plane. Therefore, within the
single-mode RTA, for the phonon modes below 5 THz, all of which belong to the
acoustic phonon branches, the vibration modes along the $a$-$b$ plane are more
easily scattered in the $\beta$ phase, no matter whether they are longitudinal
or transverse. For the panel of $\beta$-Si$_3$N$_4$ in
Fig.~\ref{fig:Fig7_338}-b, a straight line can divide the phonon modes into the
two groups. The numbers of the phonon modes in the upper and lower parts are
157 and 58, whose ratio is rational as the population ratio of the vibration
modes along and out of the $a$-$b$ plane.


\begin{figure}[ht]
 \centering
  \includegraphics[width=\linewidth]{figure_analyze_gamma3_m1010_print.eps} \caption{(color
	  online) \textcolor{blue}{Distribution of linewidths $\omega_\lambda$ $\leq$ 5 THz
		  with colormaps with respect to strengths of eigenvector components along $\mathbf q$ (a)
		  and on $a$-$b$ plane (b).} \label{fig:Fig7_338}} 
 \centering
\end{figure}

% gv weighted dos
%\begin{align}
% \label{eq:v2dos}
% \langle\text{v}^2_i(\omega)\rangle = \frac{1}{N_\mathbf{q}\Omega}
% \int_0^\omega \sum_\lambda
% \text{v}_{\lambda,i}^2\delta(\omega'-\omega)d\omega'.
%\end{align}

\section{Summary}

In the present study, we investigate the lattice thermal conductivities of the
three Si$_3$N$_4$ phases, by using the lattice dynamics based on the {\it
ab-initio} interatomic force constants. The main remarks are as follows:

1) In the $\alpha$- and $\beta$-Si$_3$N$_4$, whose crystal structures are
characterized by the stacking manners of the basal layers, which alter the
LTCs. In $\alpha$-Si$_3$N$_4$, the LTC tensors are rather isotropic, while
$\kappa$$_{zz}$ of the $\beta$ phase is much larger than the others, showing
remarkable anisotropy in the LTC tensor. 

2) In the $\alpha$ phase, the acoustic mode phonons below 6 THz are the main
heat carriers, while in the $\beta$ phase, the phonons below 12 THz contribute
to the thermal conductivity. Their group velocities are significantly different
between the phases; their linewidths are basically similar due to the similar
impacts of the phonon-phonon interaction strengths and selection rules.
Therefore the difference in the group velocities alone qualitatively explains
the difference of anisotropy.

~\textcolor{blue}{3) In the $\gamma$ phase, the frequency distribution of the
phonon mode contributions to LTC is found to be similar to that for
$\kappa$$_{xx}$ of $\beta$-Si$_3$N$_4$ although the respective phonon properties 
(group velocities and linewidths) are much different from those of the other phases}

\section*{ACKNOWLEDGMENTS}
The present work was partly supported by Grants-in-Aid for Scientific
Research of MEXT, Japan (Grant No. 15K14108 and ESISM (Elements Strategy
Initiative for Structural Materials) of Kyoto University).

\appendix
\section{Pressure dependence of LTC of $\gamma$-phase}
\begin{figure}[ht]
 \begin{center}
  \includegraphics[width=0.80\linewidth]{S1.eps} \caption{(color online)
  Pressure dependence of LTC of $\gamma$-Si$_3$N$_4$.  \label{fig:S1} }
 \end{center}
\end{figure}
\bibliography{Si3N4}
\end{document}
