%#BIBTEX bibtex Si$_3$N$_4$
\documentclass[twocolumn,amsmath,amssymb,a4paper,prb,superscriptaddress,floatfix]{revtex4-1}
\usepackage[dvipdfmx]{graphicx}
\usepackage{natbib}
\usepackage{multirow}
\usepackage{amsmath}
\usepackage{bm}
\usepackage{mathrsfs}
\usepackage{url}
\usepackage{color}
\usepackage{here}
\usepackage{ulem}
\begin{document}

\title{First-principles calculation of lattice thermal
conductivities of $\alpha$-, $\beta$-, and $\gamma$-Si$_3$N$_4$}

\author{Kazuyoshi Tatsumi} \email{k-tatsumi@imass.nagoya-u.ac.jp}
\affiliation{Institute of
Materials and Systems for Sustainability, Nagoya University, Chikusa,
Nagoya 464-8603, Japan}
\affiliation{Center for Elements Strategy
Initiative for Structural Materials, Kyoto University, Sakyo, Kyoto
606-8501, Japan}

\author{Atsushi Togo}
\affiliation{Center for Elements Strategy Initiative for Structural
Materials, Kyoto University, Sakyo, Kyoto 606-8501, Japan}

\author{Isao Tanaka}
\affiliation{Center for Elements Strategy Initiative for Structural
Materials, Kyoto University, Sakyo, Kyoto 606-8501, Japan}
\affiliation{Department of Materials Science and
Engineering, Kyoto University, Sakyo, Kyoto 606-8501, Japan}
\affiliation{Nanostructures Research Laboratory, Japan Fine Ceramics
Center, Atsuta, Nagoya 456-8587, Japan}

\begin{abstract}
The lattice thermal conductivities of $\alpha$-, $\beta$- and $\gamma$-Si$_3$N$_4$
phases are investigated from {\it ab-initio} anharmonic lattice dynamics, within
the single-mode relaxation-time approximation of the linearized phonon Boltzmann
transport equation. At 300 K, the lattice thermal conductivity of
$\beta$-Si$_3$N$_4$ is calculated as $\kappa_{xx}=73$ and $\kappa_{zz}=199$ (in
units of W m$^{-1}$ K$^{-1}$), which is consistent with the reported experimental
values of 69 and 180, respectively. For $\alpha$-Si$_3$N$_4$, $\kappa_{xx}=68$
and $\kappa_{zz}=100$ are obtained.  The difference in anisotropy between these
phases originates from the characteristic differences in their phonon band
structures, which is closely related to the crystal structures.  In $\alpha$-Si$_3$N$_4$,
acoustic-mode phonons below 6 THz are the main heat carriers, while in
$\beta$-Si$_3$N$_4$, the phonon modes up to 12 THz contribute to the lattice
thermal conductivity. In $\gamma$-Si$_3$N$_4$, $\kappa=77$ is obtained. The
distribution of phonon mode contributions to the lattice thermal conductivity with
respect to phonon frequency closely resembles that for $\kappa_{xx}$
of $\beta$-Si$_3$N$_4$, although the phonon lifetimes for $\gamma$-Si$_3$N$_4$ are
half as short as those for $\beta$-Si$_3$N$_4$.
\end{abstract}

\maketitle

\section{Introduction}
Several nitride insulators are known to exhibit high thermal conductivity, which
is important for heat transfer materials at elevated temperatures. For example,
Slack {\it et al.}\cite{slack} reported that wurtzite-type AlN has thermal
conductivity that exceeds 100 W m$^{-1}$ K$^{-1}$. Si$_3$N$_4$ has become another
promising thermal conductive insulator because its thermal conductivity has been
improved up to 177 W m$^{-1}$ K$^{-1}$ through the use of advanced ceramic technologies
related to densification and microstructure
control.\cite{zhou,hirao-rev,watari,hirosaki} The Si$_3$N$_4$ ceramics
also exhibit high mechanical strength at elevated temperatures; therefore, they are
regarded as ideal materials for use in various applications, such as engine
components, gas turbines, and heat sink substrates of power semiconductor
devices.

At atmospheric pressure, Si$_3$N$_4$ has two phases, $\alpha$ and $\beta$, which
are generally considered as low- and high-temperature phases,
respectively.\cite{zhou,hirosaki-md,riley} Their crystal structures belong to
the {\it P}31{\it c} and {\it P}6$_3${\it /m} space groups, respectively.\cite{yashima,boulay} These
structures have different manners of stacking equivalent basal layer structures
composed of SiN$_4$ tetrahedra.\cite{hampshire} Fig.~\ref{fig:Fig1_cryst}
depicts these layer structures from the principal axis direction. They are
denoted as A, B, C, and D in the $\alpha$ phase, and A and B in the $\beta$
phase. The stacking manners are thus ABCDABCD... and ABAB..., respectively. The
$\alpha$ phase has additional two layer structures of C and D, which are related
to A and B by the $c$ glide operation.\cite{hampshire} Along this direction, the
lattice constant of the $\alpha$ phase is approximately twice as long as that of
the $\beta$ phase.
\begin{figure}[ht]
 \begin{center}
  \includegraphics[width=0.90\linewidth]{Fig1_crystal_str2.pdf} \caption{(color
  online) Crystal structures of $\alpha$- and $\beta$-Si$_3$N$_4$. Stacking of
  SiN$_4$ tetrahedron layers are shown at the left. (a) ABCDABCD... for
  $\alpha$-Si$_3$N$_4$. (b) ABAB... for $\beta$-Si$_3$N$_4$.  Space group
  diagrams\cite{inttableA} for {\it P31c} ($\alpha$-Si$_3$N$_4$) and {\it
  P6$_3$/m} ($\beta$-Si$_3$N$_4$)
  are shown at the right.}
  \label{fig:Fig1_cryst} 
 \end{center}
\end{figure}

The experimental thermal conductivities
\cite{zhou,hirao-rev,watari,hirosaki,hirai} of the Si$_3$N$_4$ polymorphs were
measured for bulk polycrystalline samples. These values were significantly
affected by the lattice defects, impurities, shapes and orientations of the
constituent crystal grains;\cite{hirosaki-md} the intrinsic thermal conductivity
of defect-free Si$_3$N$_4$ has not been established. As an experimental approach
to determine this, Li {\it et al.}\cite{li} applied the high-resolution
thermoreflectance microscopy to single $\beta$-Si$_3$N$_4$ grains in a ceramic
sample. The thermal conductivity was analyzed as 69 and 180 W m$^{-1}$ K$^{-1}$
along the $a$ and $c$ axes, respectively.  These values respectively correspond
to the $xx$ and $zz$ elements of the lattice thermal conductivity tensor,
$\boldsymbol{\kappa}$. We consider the anisotropy of
$\kappa_{zz}/\kappa_{xx}\sim 3$ is relatively large. Hirosaki {\it et
al.}\cite{hirosaki-md} theoretically estimated $\boldsymbol{\kappa}$ by
application of the Green-Kubo formulation to the molecular dynamics (MD) method
with the interatomic potentials proposed by Vashishta {\it et
al.}\cite{vashishta}.  They calculated $\kappa$$_{xx}$ and $\kappa$$_{zz}$ of
$\alpha$-Si$_3$N$_4$ to be 105 and 225 W m$^{-1}$ K$^{-1}$, and those of
$\beta$-Si$_3$N$_4$ as 170 and 450 W m$^{-1}$ K$^{-1}$, respectively.  The ratio
$\kappa_{zz}/\kappa_{xx}$ in $\beta$-Si$_3$N$_4$ agreed well with the
experimental ratio; $\kappa_{xx}$ and $\kappa_{zz}$ were overestimated by more
than two times those 
of the corresponding experimental $\boldsymbol{\kappa}$ values. 

Based on first principles calculations and Boltzmann transport
theory\cite{phono3py}, Togo {\it{et al.}} recently calculated
$\boldsymbol{\kappa}$ of many polymorphs of the zincblende- and wurtzite-type
structures. Their crystal structures have stacking manners of the densest atom
planes as ABCABC... and ABAB..., respectively. The different stacking manners
merely altered $\boldsymbol{\kappa}$, the phonon linewidths and the phonon
density of states (DOS).\cite{phono3py} On the other hand, the previous MD
results indicated that the different stacking manners between the $\alpha$ and
$\beta$ phases altered $\boldsymbol{\kappa}$ significantly. This has not been
explained with respect to their phonon properties. Therefore, it is of interest
to investigate this based on the first principles anharmonic phonon calculation.

In addition to the $\alpha$ and $\beta$ phases, a cubic spinel phase
($\gamma$-Si$_3$N$_4$) is known to form upon compression and {\it{in situ}}
heating.\cite{zerr,zhang} The reported transition pressures are scattered from
10 to 36 GPa, depending on the experimental conditions.\cite{xu}  The $\gamma$
phase is experimentally quenched to atmospheric pressure and room temperature.
The thermal conductivity of the $\gamma$ phase has not been experimentally
reported, although it has been estimated by the Slack model.\cite{morelli} 

The present study aims to qualitatively elucidate the lattice thermal
conductivity tensors among the three Si$_3$N$_4$ phases by a first principles
approach.  We calculate $\boldsymbol{\kappa}$ of the $\gamma$ phase as well, for
systematic understanding. After the methodology is described, we
examine the validity of the present results; through comparison
of the calculated thermal properties with the available experimental and
theoretical references. 
The characteristic behaviors of  $\boldsymbol{\kappa}$
are then investigated in detail on the basis of the phonon band structures and
phonon lifetimes.

\section{Computational procedures}

\subsection{Lattice thermal conductivity calculation}

The lattice thermal conductivities were calculated by solving the linearized
Boltzmann transport equation (LBTE) within the single-mode relaxation time
approximation (single-mode RTA).  The harmonic phonon states and lattice
thermal conductivities were calculated with the phonopy\cite{phonopy} and
phono3py\cite{phono3py} software packages, respectively.  We also attempted the
direct-solution of LBTE\cite{chaput-direct} and give the calculated
$\boldsymbol{\kappa}$ values in the following section. The difference between
$\boldsymbol{\kappa}$ calculated by the single-mode RTA and that by the direct
solution was minor for our discussion, for which the detail is given in Sec.
IIc. Therefore, this research was limited to use the single-mode RTA to take
advantage of the closed form of $\boldsymbol{\kappa}$, which can be intuitively
understood in terms of mode-specific phonon properties.

In the following sections, we denote the phonon mode by $\lambda=(\mathbf{q},p)$
with the set of the phonon wave vector $\mathbf{q}$ and band index $p$ and
$-\lambda \equiv (-\mathbf{q},p)$.  The harmonic phonon frequency of the phonon
mode $\lambda$ is denoted by  $\omega_\lambda$. 
We use $e_\lambda^{i,j}$ to denote the eigenvector component
belonging to the $j$-th cartesian coordination of the $i$-th atom in the primitive unit cell.
The band index $p$ is set as 1, 2,... simply in ascending order
of the harmonic phonon frequency at a $\mathbf{q}$-point.  The relaxation time
due to phonon-phonon scattering was obtained as,
$\tau_{\lambda,\text{ph-ph}}=[2\Gamma_\lambda(\omega_\lambda)]^{-1}$, with
\begin{align}
 \label{eq:linewidth}
 &\Gamma_\lambda(\omega) = \frac{18\pi}{\hbar^2}
  \sum_{\lambda' \lambda''}
  \bigl|\Phi_{-\lambda\lambda'\lambda''}\bigl|^2 N_2(\mathbf{q},\omega).
\end{align}
In this equation, $\hbar$ is the reduced Planck constant and
$\Phi_{\lambda\lambda'\lambda''}$ denotes the three-phonon-scattering strength,
obtained by the usual coordinate transformation of third-order force constants
from direct space to phonon space.\cite{phono3py} The second- and third-order
real-space force constants were obtained by {\it ab initio} calculation, of
which the details are given in the next section. $N_2(\mathbf{q},\omega)$ is a
weighted joint DOS (WJDOS)\cite{phono3py},
\begin{align}
 \label{eq:jdos}
 &N_2(\mathbf{q},\omega) = N_2^{(1)}(\mathbf{q},\omega) +  N_2^{(2)}(\mathbf{q},\omega)
\end{align}
where 
\begin{eqnarray*}
	N_2^{(1)} & = & \frac{1}{N_\mathbf{q}} \sum_{\lambda'\lambda''}(n_{\lambda'}-n_{\lambda''})\Delta(-\mathbf{q} + \mathbf{q'} + \mathbf{q''}) \nonumber \\
								   & \times & [\delta(\omega + \omega_{\lambda'} - \omega_{\lambda''}) - \delta(\omega - \omega_{\lambda'} + \omega_{\lambda''})],\\
	N_2^{(2)} & = & \frac{1}{N_\mathbf{q}} \sum_{\lambda'\lambda''}(n_{\lambda'} + n_{\lambda''}+1)\Delta(-\mathbf{q} + \mathbf{q'} + \mathbf{q''}) \nonumber \\
								   & \times & \delta(\omega - \omega_{\lambda'} - \omega_{\lambda''}),
\end{eqnarray*}
with $\Delta$($\mathbf{x}$) giving 1 if $\mathbf{x}$ is a reciprocal lattice
vector, and otherwise zero. This constraint comes from the lattice
translational invariance that appears inside
$\Phi_{\lambda\lambda'\lambda''}$,\cite{phono3py} however, we let it appear in
$N_2$ in Eq.~(\ref{eq:linewidth}) in the same manner as in
Ref.~\onlinecite{sio2}, for the analysis given below. $N_\mathbf{q}$
is the number of $\mathbf{q}$-points. The weighting terms are composed of
$n_\lambda=[\exp(\hbar\omega_\lambda/\mathrm{k_B}T)-1]^{-1}$, the Bose-Einstein
distribution at temperature $T$ with $\mathrm{k_B}$ of Boltzmann constant.

To more realistically compare the  calculated $\boldsymbol{\kappa}$ with the
measured thermal conductivities, the isotopic scattering effect due to the natural isotope
distribution was taken into account according to the second-order perturbation
theory.\cite{tamura} Using the relaxation times for the phonon-phonon scattering
and isotopic scattering, $\tau_{\lambda,\text{ph-ph}}$ and
$\tau_{\lambda,\text{iso}}$, respectively, the total relaxation time for a phonon mode,
$\tau_{\lambda}$, was calculated by assuming Matthiessen's rule, 
$1/\tau_{\lambda} = 1/\tau_{\lambda,\text{ph-ph}} +
1/\tau_{\lambda,\text{iso}}$.

%The experimental thermal conductivities in the Si$_3$N$_4$ system were
%measured for polycrystalline samples and not from single
%crystals. The conductivities measured in a polycrystalline area were affected
%by various lattice defects within that area, such as grain boundaries, impurities, and
%vacancies. We crudely took them into account by the relaxation time
%$\tau_{\lambda,\text{bs}}=L/|\mathbf{v}_\lambda|$ of a phonon boundary
%scattering model, where $\mathbf{v}_\lambda = \nabla_{\mathbf{q}}\omega_\lambda$
%is the group velocity and $L$ is a parameter related to the boundary mean free
%path. We consider $\tau_{\lambda,\text{bs}}$ as a variable parameter and partly
%include it in the calculated $\boldsymbol{\kappa}$, according to Matthiessen's rule. 

The closed form of $\boldsymbol{\kappa}$ within the RTA was obtained via
\begin{align}
 \label{eq:kappa}
 \boldsymbol{\kappa} = \frac{1}{N_\mathbf{q}\Omega} \sum_\lambda
 \tau_\lambda \mathbf{v}_\lambda \otimes \mathbf{v}_\lambda c_\lambda,
\end{align}
where $\Omega$ is the unit cell volume,  $\mathbf{v}_\lambda =
\nabla_{\mathbf{q}}\omega_\lambda$ is the group velocity, and $c_\lambda=\frac {\partial (n_\lambda\hbar\omega_\lambda)}{\partial T}$ is
the mode heat capacity. To analyze $\boldsymbol{\kappa}$ in detail, the
cumulative thermal conductivity:
\begin{align}
 \label{eq:cum-kappa}
 \boldsymbol{\kappa}^\text{c}(\omega) = \frac{1}{N_\mathbf{q}\Omega}
 \int_0^\omega \sum_\lambda
 \tau_\lambda \mathbf{v}_\lambda \otimes \mathbf{v}_\lambda
 c_\lambda \delta(\omega'-\omega_{\lambda})d\omega',
\end{align}
and its derivative $\frac{d
\boldsymbol{\kappa}^\text{c}(\omega)}{d\omega}$: 
\begin{align}
 \label{eq:dcum-kappa}
 \frac{d\boldsymbol{\kappa}^\text{c}(\omega)}{d\omega} = \frac{1}{N_\mathbf{q}\Omega}
 \sum_\lambda
 \tau_\lambda \mathbf{v}_\lambda \otimes \mathbf{v}_\lambda
 c_\lambda\delta(\omega-\omega_{\lambda}),
\end{align}
were calculated to
determine the phonon mode
contributions to $\boldsymbol{\kappa}$.

\subsection{Computational details}

The force constants required were calculated using the
first-principles projector augmented wave method\cite{paw} (VASP
code\cite{vasp-1996,vasp-1995, vasp-1999}). The generalized gradient
approximation (GGA) parameterized by Perdew, Burke, and Ernzerhof\cite{pbe} was
used for the exchange correlation potential. A plane wave energy cutoff of 500
eV was employed. The crystal structures were optimized for 0 K and 0 GPa until
the residual forces acting on the constituent atoms were less than $10^{-6}$
eV \AA$^{-1}$. Here, the temperature and pressure were considered only for the
electronic system and the zero point lattice vibration was not considered. The
calculated lattice parameters were $a=7.808$ {\AA} and $c=5.659$ {\AA} for the
$\alpha$ phase, $a=7.660$ {\AA} and $c=2.925$ {\AA} for the $\beta$ phase, and
$a=7.787$ {\AA} for the $\gamma$ phase, which are in agreement with the experimental
data\cite{yashima,boulay,paszkowicz} within +0.7 \% error. The lattice volume
optimized with the local density approximation (LDA)\cite{lda} for the exchange
correlation potential was, for $\beta$-Si$_3$N$_4$, 3 \% smaller than the volume
optimized with GGA, which is a typical volume contraction of LDA. 
$\kappa_{xx}}$ and $\kappa_{zz}}$ calculated with LDA were larger by 0.3 and 2.6
\% than those calculated with GGA. For our discussion, these differences are
sufficiently small; therefore, the impact of the choice of exchange correlation
potential is considered to be minor in this study.

\begin{table}[ht]
	\caption{\label{table:LTC} Calculated lattice thermal conductivities 
 of $\alpha$-, $\beta$-, and $\gamma$-Si$_3$N$_4$
 (W K$^{-1}$ m$^{-1}$) at 300 K with respect to several combinations of
 supercell sizes.}
 \begin{ruledtabular}
  \begin{tabular}{ccccc}
   \multirow{2}{*}{Phase}
   & \multicolumn{2}{c}{Supercell (\# of atoms)} &
   \multicolumn{2}{c}{LTC} \\
   \cline{2-5}
   & $3^\text{rd}$ force constants & $2^\text{nd}$ force constants & $xx$ & $zz$ \\
   \hline
   \multirow{6}{*}{$\alpha$}
   & $1\times 1\times 1$ (28) & $1\times
   1\times 1$ (28) & 37 &   57 \\ 
   & $1\times 1\times 2$ (56) & $1\times
   1\times 2$ (56) & 41 &   79 \\ 
   & $1\times 1\times 1$ (28) & $2\times
   2\times 2$ (224) & 55 &   81 \\ 
   & $1\times 1\times 2$ (56) & $2\times
   2\times 2$ (224) & 67 &   95 \\ 
   & $1\times 1\times 2$ (56) & $2\times
   2\times 3$ (336) & 68 &  97 \\ 
   & $1\times 1\times 2$ (56) & $3\times
   3\times 4$ (1008) & 68 &  100 \\ 
   \hline
   \multirow{5}{*}{$\beta$}
   & $1\times 1\times 2$ (28) & $1\times
   1\times 2$ (28) & 44 & 173 \\ 
   & $1\times 1\times 2$ (28) & $2\times
   2\times 4$ (224) & 76 &  208 \\ 
   & $1\times 1\times 3$ (42) & $2\times
   2\times 4$ (224) & 71 & 194 \\ 
   & $1\times 1\times 3$ (42) & $2\times
   2\times 5$ (280) & 72 & 196 \\ 
   & $1\times 1\times 3$ (42) & $3\times
   3\times 8$ (1008) & 73 & 199 \\ 
   \hline
   \multirow{3}{*}{$\gamma$}
   & $1\times 1\times 1$ (56) & $1\times
   1\times 1$ (56) & \multicolumn{2}{c}{72} \\ 
   & $1\times 1\times 1$ (56) & $2\times
   2\times 2$ (448) & \multicolumn{2}{c}{77} \\ 
   & $1\times 1\times 1$ (56) & $3\times
   3\times 3$ (56) & \multicolumn{2}{c}{79} \\ 
  \end{tabular}
 \end{ruledtabular}
\end{table}

The force constants were calculated by the finite difference
approach\cite{wei-supercell}. For this calculation, the following supercells
were adopted: $1\times 1\times2$, $1\times 1\times3$, and $1\times 1\times1$
supercells of the conventional unit cells for the calculations of the
third-order force constants of $\alpha$, $\beta$, and $\gamma$-Si$_3$N$_4$,
respectively, and $3\times 3\times4$, $3\times 3\times8$ and $2\times 2\times2$
for those of the second-order force constants.  The length of the induced atomic
displacements was set to 0.03 \AA.  Table \ref{table:LTC} shows
$\boldsymbol{\kappa}$ calculated with several different sets of the supercells,
which indicates that the calculated $\boldsymbol{\kappa}$ has reasonable
convergence with respect to the size of the supercells. 

Uniform $\mathbf{k}$-point sampling meshes of $4\times 4\times 2$, $4\times
4\times 3$, and $3\times 3\times 3$ were employed for calculations of the
third-order force constants of the $\alpha$, $\beta$, and $\gamma$ phases. For
the $\alpha$ and $\beta$ phases, the center of the $a^*b^*$ plane was sampled,
while the center on the $c^*$-axis was not. For the $\gamma$ phase, a non-$\Gamma$
center mesh was used. For the calculations of the second-order force constants,
the $\Gamma$-point was only sampled for the $\alpha$ and $\beta$
phases, and the
only one $\mathbf{k}=(0.5, 0.5, 0.5)$ point was sampled for the $\gamma$ phase.
The $\mathbf{q}$-point sampling meshes of $10\times 10\times 14$, $10\times
10\times 26$, and $12\times 12\times 12$ were employed to calculate
$\boldsymbol{\kappa}$ in Eq.~(\ref{eq:kappa}) for the $\alpha$, $\beta$, and
$\gamma$ phases, respectively.

Non-analytical term correction\cite{wang} was applied to the second-order force
constants to take into account the long range coulombic forces present in ionic
crystals. For the correction, static dielectric constants and Born effective
charges were calculated using the density functional perturbation theory
as implemented in the VASP code\cite{vasp-lepsiron,lepsiron}.

The effect of lattice thermal expansion on $\boldsymbol{\kappa}$ was examined
by the calculation of $\boldsymbol{\kappa}$ for several finite
temperatures with the crystal structures optimized for the
corresponding temperatures within the quasi-harmonic approximation
(QHA)\cite{dove-p76}. These $\boldsymbol{\kappa}$ were different from those
calculated for the same temperatures with the structure
optimized for 0 K. We consider these differences as the effect of lattice
thermal expansion. The differences in $\boldsymbol{\kappa}$ for $T$=300, 600,
900, 1200, and, 1500 K, for the $\beta$ phase, were within 1 \%. They 
were similar to those for Si and Ge calculated by Ward {\it{et
al.}}\cite{ward-ltc}.
For the present study, these differences are negligible and
for finite temperatures $\boldsymbol{\kappa}$ calculated with the
structure optimized for 0 K was adopted.


The volumetric thermal expansion coefficients were also calculated. 
Comparison with the experimental coefficient is useful to validate the present
thermal conductivity calculation because both the thermal expansion
and $\boldsymbol{\kappa}$ originate from the anharmonicity of the interatomic
potential.
The calculated coefficients of the $\alpha$, $\beta$, and $\gamma$ phases were
4.31$\times 10^{-6}$,  4.19$\times 10^{-6}$, and 1.13$\times 10^{-5}$
K$^{-1}$
for 300 K, while the experimental values\cite{minikayev-alpha, gamma-expand}
were 3.75$\times 10^{-6}$ , 3.55$\times 10^{-6}$, and 9.48$\times
10^{-6}$ K$^{-1}$. The calculation systematically overestimated the experimental
values, but reproduced the experimental tendencies, including that the $\alpha$
phase has a slightly larger thermal expansion coefficient than the $\beta$
phase. This supports the validity of the present calculation to qualitatively
compare the calculated $\boldsymbol{\kappa}$ among the Si$_3$N$_4$ phases.

To compare the microscopic phonon properties among the three phases under the
same conditions, the results calculated at 0 GPa are shown and discussed.  For
the $\gamma$ phase, this means that we assume the condition of a virtually
quenched $\gamma$ phase at 0 GPa from the high pressure. To examine the
analytical continuity of the properties with respect to pressure,
$\boldsymbol{\kappa}$ of the $\gamma$ phase was calculated at 10, 20, and 40
GPa, as shown in Fig.~\ref{fig:S1}. The phenomenological behavior of the linear
dependence of $\boldsymbol{\kappa}$ with respect to the pressure was reproduced,
similar to that in Ref.~\onlinecite{andersson-pressure}. The slope was 2.89
W m$^{-1}$ K$^{-1}$ GPa$^{-1}$ for the $\gamma$ phase.  From this dependence, we
consider that the microscopic values are also varied smoothly with the pressure
and those at 0 GPa are valuable for comparison with the corresponding values of
the $\alpha$ and $\beta$ phases.

\subsection{Direct solution of LBTE}

The advantage of employing the single-mode RTA for thermal conductivity
calculations is the closed form Eq.~(\ref{eq:kappa}), by which the qualitative character of
$\boldsymbol{\kappa}$ can be intuitively understood in terms of the phonon-mode
specific properties. The microscopic understanding of the full solution of LBTE
is still under development,\cite{cepellotti-relaxons} and the microscopic
picture based on collective phonons\cite{hardy-collective} will require more
complicated investigation.

Single-mode RTA solutions of LBTE often underestimate the full
solution.\cite{mukhopadhyay-ltc,ward-ltc} To check this underestimation,
$\boldsymbol{\kappa}$ for the $\alpha$ and $\beta$ phases were calculated by the
direct solution of LBTE\cite{chaput-direct}, which is one of the methods of LBTE
full solutions. $\kappa_{xx}$ and $\kappa_{zz}$ without the isotope effect were
69 and 102 W m$^{-1}$ K$^{-1}$ for the $\alpha$ phase, and 76 and 238
W m$^{-1}$ K$^{-1}$ for the $\beta$ phase, respectively, while the corresponding
single-mode RTA values were 70 and 102 W m$^{-1}$ K$^{-1}$ for the $\alpha$ phase,
and 76 and 210 W m$^{-1}$ K$^{-1}$ for the $\beta$ phase. $\kappa_{zz}$ for the
$\beta$ phase from the direct solution was 13 \% larger than that of the
single-mode RTA solution. The differences in $\boldsymbol{\kappa}$ between the
LBTE solutions are not significant; therefore, we expect that the physics of
these lattice thermal conductivities is well understood within the
single-mode RTA at the current level of our interest. Therefore, we discuss the
lattice thermal conductivities calculated by the single-mode RTA solution.

\section{Results and discussion}

\subsection{Lattice thermal conductivities}

\begin{table}[ht]
 \caption{\label{table:LTC-exp} Calculated thermal conductivities of
 $\alpha$-Si$_3$N$_4$ (trigonal), $\beta$-Si$_3$N$_4$ (trigonal), and
 $\gamma$-Si$_3$N$_4$ (cubic) at 300
 K in units of W m$^{-1}$ K$^{-1}$, compared with the experimental and theoretical
 reference data, and bulk moduli $B$ (in
 units of GPa), calculated from the elastic constant calculation routine\cite{elastic} in the VASP code.}

\begin{ruledtabular}
 \begin{tabular}{ccccccccc}
   & \multicolumn{3}{c}{This work} & \multicolumn{3}{c}{Ref. Theo.}
   & \multicolumn{2}{c}{Ref. Expt.} \\
   \cline{2-9}
   & $\kappa_{xx}$ & $\kappa_{zz}$ & $B$ & $\kappa$ & $\kappa_{xx}$ & $\kappa_{zz}$ & $\kappa_{xx}$ & $\kappa_{zz}$ \\
   \hline
   $\alpha$-Si$_3$N$_4$ & 68 & 100 & 224 & 70\footnotemark[1] & 105\footnotemark[2] & 225\footnotemark[2] & - & -  \\
   $\beta$-Si$_3$N$_4$ & 73 & 199 & 237 & 250\footnotemark[1] & 170\footnotemark[2] & 450\footnotemark[2] & 69\footnotemark[3] & 180\footnotemark[3] \\
   $\gamma$-Si$_3$N$_4$ & 77 & - & 296 & 80\footnotemark[1] & - & - & - & - 
   \footnotetext[1]{Ref.~\onlinecite{morelli}, Slack model.}
   \footnotetext[2]{Ref.~\onlinecite{hirosaki-md}, molecular dynamics (Green-Kubo).}
   \footnotetext[3]{Ref.~\onlinecite{li}, single crystalline grains of poly-crystals.}
  \end{tabular}
 \end{ruledtabular}
\end{table}

\begin{figure}[H]
	 \begin{center}
		   \includegraphics[width=0.90\linewidth]{Figure_kaccum_for_alpha_beta_gamma_phases.pdf}
		   \caption{Cumulative thermal conductivity of the three phases. 
		   \label{fig:kaccum} }
    \end{center}
\end{figure}
Table~\ref{table:LTC-exp} shows the calculated
$\boldsymbol{\kappa}$ for 300 K with the isotope effect. $\beta$-Si$_3$N$_4$ has a markedly more
anisotropic $\boldsymbol{\kappa}$ than $\alpha$-Si$_3$N$_4$.  The directional
averages $\sum_i \kappa_{ii}/3$  are 79, 115,  and 77 W m$^{-1}$ K$^{-1}$ for the
$\alpha$, $\beta$, and $\gamma$ phases, respectively. The value for the
$\gamma$ phase is similar to that for the $\alpha$ phase, despite the
comparatively large difference among the bulk moduli ($B$) that are also shown
in Table~\ref{table:LTC-exp}.   

Table~\ref{table:LTC-exp} also lists the previously reported
experimental\cite{li} and theoretical\cite{hirosaki-md} $\boldsymbol{\kappa}$
for reference. The theoretical results\cite{morelli} of the Slack model, which
do not include the anisotropy in $\boldsymbol{\kappa}$, are shown as $\kappa$ in
Table~\ref{table:LTC-exp}. Compared to the $\boldsymbol{\kappa}$ from
MD\cite{hirosaki-md}, our $\boldsymbol{\kappa}$ for the $\beta$ phase has better
agreement with the experimental $\boldsymbol{\kappa}$.  Compared to $\kappa$
from the Slack model, our directional average $\sum_i \kappa_{ii}/3$ is also
much closer to the experimental average. 

Fig.~\ref{fig:kaccum} shows the cumulative thermal
conductivity, $\boldsymbol{\kappa}^c(\omega)$. 
From this figure, it is evident that in the
$\alpha$, $\beta$, and $\gamma$ phases, the phonon modes with their frequencies
up to $\sim$ 6, 12 and 10 THz largely contribute to each respective
$\boldsymbol{\kappa}$. 

\subsection{Distribution of group velocity in Brillouin zone}

The Brillouin zones and phonon band diagrams of the three phases are shown in
Fig.~\ref{fig:Fig4_ver5_338}(a). In this figure, we investigate the frequency
gradients, the group velocities projected on the paths along the nonequivalent
axes of the reciprocal lattice. We particularly focus on the anisotropy of the
group velocities  in the $\alpha$ and $\beta$ phases. This was not investigated
in the previous works. The band diagrams on the other high-symmetry paths are
almost identical to those reported\cite{kuwabara,xu} and thus are not shown.
For the $\alpha$ and $\beta$ phases, in order to investigate the behaviors of
the group velocities at the other $\mathbf{q}$ points, the cross-section of the
phonon frequency  
distribution on the $b^*c^*$ plane is shown in Fig.~\ref{fig:Fig4_ver5_338}(b). Since the
cross-section on the other planes 
containing the $c^*$ axis did not differ much from that shown in 
Fig.~\ref{fig:Fig4_ver5_338}(b), we focus on the $b^*c^*$
plane as a representative of all such planes. We show only the frequencies of the 
four modes from the lowest frequency, i.e., $p$=1, 2, 3, and 4,  because they contribute significantly to
$\boldsymbol{\kappa}$.
The A$\Gamma$ and $\Gamma$K paths of the band diagram in
Fig.~\ref{fig:Fig4_ver5_338}(a) respectively correspond to the left and bottom
edges of the cross-section in Fig.~\ref{fig:Fig4_ver5_338}(b).
In order to show more clearly the relationship between the band diagram and the
cross-section, in the band diagram, a thick gray line is ovelayed for the dispersion curves
corresponding to the profiles of the contour map for $p$=4 on the left and bottom edges. 

In the band diagram of the $\alpha$ phase, the acoustic
phonon modes show similar $\omega_\lambda$ values at the A and K points, 
indicating the isotropic group velocities of the acoustic phonons.
The contour maps including the most of the acoustic phonon modes are those for $p$=1, 2, and
3. For the $\alpha$ phase, these maps clearly show the
isotropic group velocities: The most of the iso-frequency lines in the contours
are parallel to the circular dash lines inserted as a guide.  In contrast, in
the band diagram of the $\beta$ phase, the acoustic phonon modes at the A point show 
much higher $\omega_\lambda$ than those at the K point.
In the contour maps of $p$=1, 2, and 3 for the $\beta$ phase, the gradients of

\begin{figure}[H]
	 \begin{center}
		   \includegraphics[width=0.90\linewidth]{Fig4_ver5_338_resize2_woDOS_color_simplified_no_symm.pdf}
		   \caption{(color online) (a) Brillouin-zones and  band diagrams of the three phases. (b) Contour maps 
			   of phonon frequencies for the $\alpha$ and $\beta$ phases, on the  $b^*c^*$  planes of Brillouin-zones. 
			   The maps for the four lowest-frequency phonon modes ($p$=1, 2, 3,
			   and 4) are shown. In the band diagrams for the $\alpha$ and
			   $\beta$ phases, the dispersion curve for $p$=4 is denoted by a thick gray line.
		   \label{fig:Fig4_ver5_338} }
    \end{center}
\end{figure}
\noindent
$\omega_\lambda$ are, in the most part, parallel to the vertical edge; the
group velocities orient to the $c^*$ axis direction.
This difference between the $\alpha$ and $\beta$ phases is
due to the $\Gamma$-A path lengths. The $\beta$ phase has an approximately twice
longer path than the $\alpha$ phase; the lattice constant $c$ of the $\beta$
phase is nearly half that of the $\alpha$ phase, owing to the different stacking
manners of the basal layer structures (Fig.~\ref{fig:Fig1_cryst}). 
Comparing the contour maps for $p$=4 between the $\alpha$ and $\beta$ phases, there is a
tendency that $\beta$ phase has larger frequency gradients than $\alpha$ phase.
Comparing the band diagrams of the $\alpha$ and $\beta$ phases, the same
tendency is seen for the most of the other optical phonon modes.

In the $\gamma$ phase, the dispersion curve of the longitudinal acoustic phonon
modes is almost linear. Their frequencies near the L point is much higher
than the longitudinal acoustic phonon frequencies near the A or K points of the
$\alpha$ and $\beta$ phases.  The gradients around the $\Gamma$ point are
the largest among the three phases, as expected by the largest $B$.

\onecolumngrid


\begin{figure*}[h]
% \begin{center}
	 \includegraphics[width=0.9\linewidth]{figure_dos_jdos_dkc_m1010_dos_wo_arrows_bw_nolog.pdf}
  \caption{Microscopic phonon properties of three Si$_3$N$_4$
	  phases. (a) Cumulative thermal conductivity $\boldsymbol{\kappa}^\text{c}$ and
	  its frequency derivative
	  , (b) DOS as $g(\omega)$, (c) DOS weighted with $\mathbf{v}_\lambda \otimes
	  \mathbf{v}_\lambda$ as $\boldsymbol{h}(\omega)$, and (d) scatter plots of phonon
	  lifetimes and phonon frequencies, $(\tau_\lambda,\omega_\lambda)$.
  \label{fig:Fig5_338_rev} }
% \end{center}
\end{figure}
\twocolumngrid

\subsection{Frequency distributions of phonon properties}

In the previous section, we have investigated the anisotropy in
$\mathbf{v}_\lambda$, which may explain the anisotropy in $\boldsymbol{\kappa}$.
Here we examine which terms in Eq.(\ref{eq:kappa}) characterize the behavior of
the calculated $\boldsymbol{\kappa}$. In the following, we omit the term of mode
heat capacity because it is approximately constant for the phonon modes that
mainly carry heat at 300 K.  For simplicity, the effect of isotope scattering is
not considered in this section. For the investigation, the frequency derivative of
cumulative thermal conductivity, $d\boldsymbol{\kappa}^c/d\omega$ in
Eq.(\ref{eq:dcum-kappa}), is shown at the top of Fig.~\ref{fig:Fig5_338_rev}. 

Assuming that $\tau_\lambda$ and $\mathbf{v}_\lambda$ are constant, then
$d\kappa_{ii}^c/d\omega$ ($ii$=$xx,zz$) are proportional to the phonon
DOS: 
\begin{align}
 \label{eq:dos}
 g(\omega) = \frac{1}{N_\mathbf{q}}
 \sum_\lambda
 \delta(\omega-\omega_{\lambda}).
\end{align}
We refer to $g(\omega)/\Omega$ as frequency distributions of heat
carrier density. Alternatively, assuming that only $\tau_\lambda$ is constant,
then $d\boldsymbol{\kappa}^c/d\omega$ is proportional to:
\begin{align}
 \label{eq:wdos}
 \boldsymbol{h}(\omega) = \frac{1}{N_\mathbf{q}\Omega}
 \sum_\lambda
 \mathbf{v}_\lambda \otimes \mathbf{v}_\lambda
 \delta(\omega-\omega_{\lambda}),
\end{align}
from which we examine the impacts of both of $\mathbf{v}_\lambda$ and the heat carrier
density. $g(\omega)/\Omega$ and  $\boldsymbol{h}(\omega)$
are shown in
Figs.~\ref{fig:Fig5_338_rev}(b) and (c). As for the frequency
variation of $\tau_{\lambda,\text{ph-ph}}$, the phonon lifetimes are shown as scatter
plots of $(\tau_{\lambda,\text{ph-ph}},\omega_\lambda)$ in Fig.~\ref{fig:Fig5_338_rev}(d).

Comparison of the $\alpha$ and $\beta$ phases indicates their phonon lifetimes
distributions are qualitatively similar, except for a striking difference below
$\sim$ 5 THz, which will be examined later. The markedly different
$d\kappa_{ii}^c/d\omega$ between the two phases are therefore ascribed to the
corresponding $h_{ii}$. The overall spectral shapes of $g(\omega)/\Omega$ are also
similar between the two phases; therefore, $\mathbf{v}_\lambda$ alone accounts
for the different behavior of $d\kappa_{ii}^c/d\omega$. It is thus concluded
that the different anisotropy in $\boldsymbol{\kappa}$ can be qualitatively
explained by the different $\mathbf{v}_\lambda$.
In contrast, for the zincblende and wurtzite
structures, the group velocities are suggested to be similar from their band
structures\cite{phono3py}.
This must result in similar $\boldsymbol{\kappa}$ between these
structures, irrespective of the stacking manner. 

The $\gamma$ phase has much different $g(\omega)/\Omega$, $\boldsymbol{h}(\omega)$,
and, $\tau_{\lambda,\text{ph-ph}}$ from the other phases, as expected from the
large differences in their crystal structures. The most significant difference
is in the phonon lifetimes. For 4 THz $\lesssim\omega_\lambda\lesssim$ 10 THz,
the phonon lifetimes are approximately half as short as those of the other
phases. We will examine this in detail later. As a result,
$d\kappa_{xx}^c/d\omega$ has relatively low intensities. The longitudinal
acoustic phonon branch increases its frequencies much significantly, as we have
examined in the band diagram; therefore, $d\kappa_{xx}^c/d\omega$ rather
gradually attenuates as the frequency increases, occasionally resembling
$d\kappa_{xx}^c/d\omega$ of the $\beta$ phase.

The distribution of phonon lifetimes is qualitatively similar between the $\alpha$ and
$\beta$ phases, although their group velocities have marked differences. This
remains a curiosity. 
Recalling Eq.~(\ref{eq:linewidth}), $\tau_{\lambda,\text{ph-ph}}$ in the
present form is dependent on WJDOS
and $|\Phi_{\lambda\lambda'\lambda''}|^2$. We examine these terms one-by-one.

\begin{figure}[ht]
 \centering
  \includegraphics[width=0.9\linewidth]{figure_wjdoss_gray.pdf} \caption{
	  WJDOS of $\alpha$- and $\beta$-Si$_3$N$_4$ at different $\mathbf
	  q$ points and that of $\gamma$-Si$_3$N$_4$ at the $\Gamma$ point. 
  The WJDOS for the $\alpha$ and $\beta$ phases in the  first and fourth rows
  are calculated at the same $\Gamma$-point but 
  with the polarization for the non-analytic term correction set along $c^*$ and
  $b^*$, respectively. \label{fig:Fig6_338} }
 \centering
\end{figure}

The frequency profiles of WJDOS in Fig.~\ref{fig:Fig6_338} are very similar between the $\alpha$ and
$\beta$ phases, for each different $\mathbf{q}$-points.  Their intensities are
scaled with $Z^2$ of which $Z$ is the number of formula units in the primitive
unit cell, to compare WJDOS for structures with different $Z$.  These profiles
show weak $\mathbf{q}$-point dependences.  The frequency profile for the
$\gamma$ phase is only shown at $\mathbf{q}=(0,0,0)$ because of the different
shape of the Brillouin zone from those in the other phases.  We checked that
the $\mathbf{q}$ dependence of WJDOS for the $\gamma$ phase was as weak as
those shown in Fig.~\ref{fig:Fig6_338} for the $\alpha$ and $\beta$ phases.
The intensities of WJDOS below $\sim$ 10 THz in the $\gamma$ phase are slightly smaller than those in
the other phases. 


\begin{table}[ht]
	\caption{\label{table:aveavepp} Averages of
	$|\Phi_{\lambda\lambda'\lambda''}|^2$ over frequency ranges of
	$\omega_\lambda$ (0--15 and 0--35 THz) and all ($\lambda'$,$\lambda'$). The
	values are in units of meV$^2$.}
 \begin{ruledtabular}
  \begin{tabular}{cccc}
	  \multirow{2}{*}{Frequency range (THz)}
   & \multicolumn{3}{c}{Phase}  \\
   \cline{2-4}
   & $\alpha$ & $\beta$ & $\gamma$ \\
   \hline
   \multirow{1}{*}{0--15}
   & 0.47 &  0.46 & 1.02 &    
   \multirow{1}{*}{0--35}
   & 2.30 & 2.30 & 2.02 &     
  \end{tabular}
 \end{ruledtabular}
\end{table}

As for $|\Phi_{\lambda\lambda'\lambda''}|^2$, in Table.~\ref{table:aveavepp},
they are averaged over two frequency ranges of 0--15 or 0--35 THz for
$\omega_\lambda$ and all indices in $\lambda'$ and $\lambda''$.  The frequency
ranges for $\omega_\lambda$ were set so that the narrower frequency range
approximately corresponds to the range where the phonon modes largely
contribute to $\boldsymbol{\kappa}$. A small change in the frequency range by a
few terahertz did not change the qualitative characteristics of the averages.
To compare the values of the structures with different $Z$, we multiply the
average by $(3n_a)^2$ where $n_a$ is the number of atoms in the primitive unit
cell. This is the same scaling manner as in Ref.~\onlinecite{sio2}. The averages are very similar for the $\alpha$ and $\beta$ phases. With
the similar impact of the WJDOS and $|\Phi_{\lambda\lambda'\lambda''}|^2$, the
phonon lifetimes in these phases are also similar. For the $\gamma$ phase, the
short $\tau_\lambda$ are attributed to the large
$|\Phi_{\lambda\lambda'\lambda''}|^2$. 



\begin{figure}[ht]
	 \centering
	   \includegraphics[width=\linewidth]{figure_analyze_gamma3_m1010_nolog_gray.pdf} \caption{(color
		   	  online) Distribution of phonon lifetimes for $\omega_\lambda$ $\leq$ 5 THz
	  		  shown in color with respect to the fraction of the eigenvector
			  component (a) along $\mathbf q$ 
		  	  and (b) parallel to the $ab$ plane (b).} \label{fig:Fig7_338} 
	   \centering
\end{figure}

Finally, we examine the striking difference in Fig.~\ref{fig:Fig5_338_rev}(d)
below $\sim$ 5 THz between the $\alpha$ and $\beta$ phases: In the $\alpha$
phase, $\tau_{\lambda,\text{ph-ph}}$ below $\sim$ 5 THz are aligned on a single
smooth line, while for the $\beta$ phase, they are scattered largely.
Fig.~\ref{fig:Fig7_338} enlarges the distributions.
In Fig.~\ref{fig:Fig7_338}(a), each plot is shown with a color
specified by $\sum_{i,j}[\frac{e_j(i,\lambda)q_j}{|\mathbf q|}]^2$, the fraction
of the eigenvector component along $\mathbf{q}$, with $q_j$ of the $j$-th cartesian coordination of $\mathbf{q}$.
The brighter (darker) color
indicates more (less) longitudinal character of the phonon mode.  In
Fig.~\ref{fig:Fig7_338}(b), the color contrast is set according to
$\sum_{i}[e_x(i,\lambda)^2+e_y(i,\lambda)^2]$, the fraction of the eigenvector
component parallel to the $ab$ plane. 

In Fig.~\ref{fig:Fig7_338}(b) for the $\beta$ phase, clearly, a shorter
lifetime at a similar frequency show a brighter color. 
Therefore, in the $\beta$
phase,  a phonon mode of the vibration within the $ab$ plane is more hindered by
the other phonons than a phonon mode of the vibration 
perpendicular to the plane: This tendency is independent on how much
longitudinal character the mode has.

A straight line is inserted in the graph for $\beta$ phase in
Fig.~\ref{fig:Fig7_338}(b).
The numbers of the phonon modes above and
below the line are 145 and 67, whose ratio is reasonable as the population ratio
between the phonon modes whose atomic vibration is parallel to the $ab$ plane
and those whose atomic vibration is perpendicular to the $ab$ plane.


\section{Summary}

In the present study, the lattice thermal conductivities of the
three Si$_3$N$_4$ phases were investigated using lattice dynamics based on the
first-principles interatomic force constants. The main remarks are as follows:

1) In $\alpha$- and $\beta$-Si$_3$N$_4$, of which the crystal structures are
characterized by the stacking manner of the basal layer structures,
$\boldsymbol{\kappa}$ is largely altered due to the differences in the harmonic
band structures induced by the different stacking manners. This is in contrast
with the zincblende and wurtzite structures in the previous
study\cite{phono3py}. $\boldsymbol{\kappa}$ for $\alpha$-Si$_3$N$_4$ is rather
isotropic, while $\kappa$$_{zz}$ for the $\beta$ phase is twice or more larger
than the other $\kappa_{ii}$ of the three phases.

2) In the $\alpha$ phase, the acoustic mode phonons below 6 THz are the main
heat carriers, while in the $\beta$ phase, the phonons below 12 THz contribute
to $\boldsymbol{\kappa}$. Their group velocities are confirmed to characterize
the behavior of $\boldsymbol{\kappa}$.

3) In the $\gamma$ phase, the frequency distribution of the phonon mode
contributions to $\boldsymbol{\kappa}$ is similar to that for $\kappa_{xx}$ of
$\beta$-Si$_3$N$_4$, which is attributed to its large phonon-phonon scattering
strength and steep longitudinal acoustic branches.



\section*{ACKNOWLEDGMENTS}
The present work was partly supported by a Grant-in-Aid for Scientific
Research (No. 15K14108 from the Ministry of Education, Culture, Sports, Science
and Technology (MEXT) Japan and the Elements Strategy Initiative for Structural
Materials (ESISM) of Kyoto University.

\appendix
\section{Dependence of the lattice thermal conductivity of $\gamma$-phase on
pressure}
\begin{figure}[ht]
 \begin{center}
  \includegraphics[width=0.80\linewidth]{S1.eps} \caption{(color online)
  Lattice thermal conductivity of $\gamma$-Si$_3$N$_4$ as a function of pressure.  \label{fig:S1} }
 \end{center}
\end{figure}
\bibliography{Si3N4}
\end{document}
